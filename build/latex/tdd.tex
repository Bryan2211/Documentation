% Generated by Sphinx.
\def\sphinxdocclass{report}
\documentclass[a4paper,10pt,french]{sphinxmanual}
\usepackage[utf8]{inputenc}
\DeclareUnicodeCharacter{00A0}{\nobreakspace}
\usepackage{cmap}
\usepackage[T1]{fontenc}
\usepackage[francais]{babel}
\usepackage{times}
\usepackage[Sonny]{fncychap}
\usepackage{longtable}
\usepackage{sphinx}
\usepackage{multirow}


\title{Développement du tableau de bord professeur}
\date{30 mars 2015}
\release{Collège du Sud}
\author{Bryan Oberson}
\newcommand{\sphinxlogo}{}
\renewcommand{\releasename}{Travail de maturité, }
\makeindex

\makeatletter
\def\PYG@reset{\let\PYG@it=\relax \let\PYG@bf=\relax%
    \let\PYG@ul=\relax \let\PYG@tc=\relax%
    \let\PYG@bc=\relax \let\PYG@ff=\relax}
\def\PYG@tok#1{\csname PYG@tok@#1\endcsname}
\def\PYG@toks#1+{\ifx\relax#1\empty\else%
    \PYG@tok{#1}\expandafter\PYG@toks\fi}
\def\PYG@do#1{\PYG@bc{\PYG@tc{\PYG@ul{%
    \PYG@it{\PYG@bf{\PYG@ff{#1}}}}}}}
\def\PYG#1#2{\PYG@reset\PYG@toks#1+\relax+\PYG@do{#2}}

\expandafter\def\csname PYG@tok@gd\endcsname{\def\PYG@tc##1{\textcolor[rgb]{0.63,0.00,0.00}{##1}}}
\expandafter\def\csname PYG@tok@gu\endcsname{\let\PYG@bf=\textbf\def\PYG@tc##1{\textcolor[rgb]{0.50,0.00,0.50}{##1}}}
\expandafter\def\csname PYG@tok@gt\endcsname{\def\PYG@tc##1{\textcolor[rgb]{0.00,0.27,0.87}{##1}}}
\expandafter\def\csname PYG@tok@gs\endcsname{\let\PYG@bf=\textbf}
\expandafter\def\csname PYG@tok@gr\endcsname{\def\PYG@tc##1{\textcolor[rgb]{1.00,0.00,0.00}{##1}}}
\expandafter\def\csname PYG@tok@cm\endcsname{\let\PYG@it=\textit\def\PYG@tc##1{\textcolor[rgb]{0.25,0.50,0.56}{##1}}}
\expandafter\def\csname PYG@tok@vg\endcsname{\def\PYG@tc##1{\textcolor[rgb]{0.73,0.38,0.84}{##1}}}
\expandafter\def\csname PYG@tok@m\endcsname{\def\PYG@tc##1{\textcolor[rgb]{0.13,0.50,0.31}{##1}}}
\expandafter\def\csname PYG@tok@mh\endcsname{\def\PYG@tc##1{\textcolor[rgb]{0.13,0.50,0.31}{##1}}}
\expandafter\def\csname PYG@tok@cs\endcsname{\def\PYG@tc##1{\textcolor[rgb]{0.25,0.50,0.56}{##1}}\def\PYG@bc##1{\setlength{\fboxsep}{0pt}\colorbox[rgb]{1.00,0.94,0.94}{\strut ##1}}}
\expandafter\def\csname PYG@tok@ge\endcsname{\let\PYG@it=\textit}
\expandafter\def\csname PYG@tok@vc\endcsname{\def\PYG@tc##1{\textcolor[rgb]{0.73,0.38,0.84}{##1}}}
\expandafter\def\csname PYG@tok@il\endcsname{\def\PYG@tc##1{\textcolor[rgb]{0.13,0.50,0.31}{##1}}}
\expandafter\def\csname PYG@tok@go\endcsname{\def\PYG@tc##1{\textcolor[rgb]{0.20,0.20,0.20}{##1}}}
\expandafter\def\csname PYG@tok@cp\endcsname{\def\PYG@tc##1{\textcolor[rgb]{0.00,0.44,0.13}{##1}}}
\expandafter\def\csname PYG@tok@gi\endcsname{\def\PYG@tc##1{\textcolor[rgb]{0.00,0.63,0.00}{##1}}}
\expandafter\def\csname PYG@tok@gh\endcsname{\let\PYG@bf=\textbf\def\PYG@tc##1{\textcolor[rgb]{0.00,0.00,0.50}{##1}}}
\expandafter\def\csname PYG@tok@ni\endcsname{\let\PYG@bf=\textbf\def\PYG@tc##1{\textcolor[rgb]{0.84,0.33,0.22}{##1}}}
\expandafter\def\csname PYG@tok@nl\endcsname{\let\PYG@bf=\textbf\def\PYG@tc##1{\textcolor[rgb]{0.00,0.13,0.44}{##1}}}
\expandafter\def\csname PYG@tok@nn\endcsname{\let\PYG@bf=\textbf\def\PYG@tc##1{\textcolor[rgb]{0.05,0.52,0.71}{##1}}}
\expandafter\def\csname PYG@tok@no\endcsname{\def\PYG@tc##1{\textcolor[rgb]{0.38,0.68,0.84}{##1}}}
\expandafter\def\csname PYG@tok@na\endcsname{\def\PYG@tc##1{\textcolor[rgb]{0.25,0.44,0.63}{##1}}}
\expandafter\def\csname PYG@tok@nb\endcsname{\def\PYG@tc##1{\textcolor[rgb]{0.00,0.44,0.13}{##1}}}
\expandafter\def\csname PYG@tok@nc\endcsname{\let\PYG@bf=\textbf\def\PYG@tc##1{\textcolor[rgb]{0.05,0.52,0.71}{##1}}}
\expandafter\def\csname PYG@tok@nd\endcsname{\let\PYG@bf=\textbf\def\PYG@tc##1{\textcolor[rgb]{0.33,0.33,0.33}{##1}}}
\expandafter\def\csname PYG@tok@ne\endcsname{\def\PYG@tc##1{\textcolor[rgb]{0.00,0.44,0.13}{##1}}}
\expandafter\def\csname PYG@tok@nf\endcsname{\def\PYG@tc##1{\textcolor[rgb]{0.02,0.16,0.49}{##1}}}
\expandafter\def\csname PYG@tok@si\endcsname{\let\PYG@it=\textit\def\PYG@tc##1{\textcolor[rgb]{0.44,0.63,0.82}{##1}}}
\expandafter\def\csname PYG@tok@s2\endcsname{\def\PYG@tc##1{\textcolor[rgb]{0.25,0.44,0.63}{##1}}}
\expandafter\def\csname PYG@tok@vi\endcsname{\def\PYG@tc##1{\textcolor[rgb]{0.73,0.38,0.84}{##1}}}
\expandafter\def\csname PYG@tok@nt\endcsname{\let\PYG@bf=\textbf\def\PYG@tc##1{\textcolor[rgb]{0.02,0.16,0.45}{##1}}}
\expandafter\def\csname PYG@tok@nv\endcsname{\def\PYG@tc##1{\textcolor[rgb]{0.73,0.38,0.84}{##1}}}
\expandafter\def\csname PYG@tok@s1\endcsname{\def\PYG@tc##1{\textcolor[rgb]{0.25,0.44,0.63}{##1}}}
\expandafter\def\csname PYG@tok@gp\endcsname{\let\PYG@bf=\textbf\def\PYG@tc##1{\textcolor[rgb]{0.78,0.36,0.04}{##1}}}
\expandafter\def\csname PYG@tok@sh\endcsname{\def\PYG@tc##1{\textcolor[rgb]{0.25,0.44,0.63}{##1}}}
\expandafter\def\csname PYG@tok@ow\endcsname{\let\PYG@bf=\textbf\def\PYG@tc##1{\textcolor[rgb]{0.00,0.44,0.13}{##1}}}
\expandafter\def\csname PYG@tok@sx\endcsname{\def\PYG@tc##1{\textcolor[rgb]{0.78,0.36,0.04}{##1}}}
\expandafter\def\csname PYG@tok@bp\endcsname{\def\PYG@tc##1{\textcolor[rgb]{0.00,0.44,0.13}{##1}}}
\expandafter\def\csname PYG@tok@c1\endcsname{\let\PYG@it=\textit\def\PYG@tc##1{\textcolor[rgb]{0.25,0.50,0.56}{##1}}}
\expandafter\def\csname PYG@tok@kc\endcsname{\let\PYG@bf=\textbf\def\PYG@tc##1{\textcolor[rgb]{0.00,0.44,0.13}{##1}}}
\expandafter\def\csname PYG@tok@c\endcsname{\let\PYG@it=\textit\def\PYG@tc##1{\textcolor[rgb]{0.25,0.50,0.56}{##1}}}
\expandafter\def\csname PYG@tok@mf\endcsname{\def\PYG@tc##1{\textcolor[rgb]{0.13,0.50,0.31}{##1}}}
\expandafter\def\csname PYG@tok@err\endcsname{\def\PYG@bc##1{\setlength{\fboxsep}{0pt}\fcolorbox[rgb]{1.00,0.00,0.00}{1,1,1}{\strut ##1}}}
\expandafter\def\csname PYG@tok@kd\endcsname{\let\PYG@bf=\textbf\def\PYG@tc##1{\textcolor[rgb]{0.00,0.44,0.13}{##1}}}
\expandafter\def\csname PYG@tok@ss\endcsname{\def\PYG@tc##1{\textcolor[rgb]{0.32,0.47,0.09}{##1}}}
\expandafter\def\csname PYG@tok@sr\endcsname{\def\PYG@tc##1{\textcolor[rgb]{0.14,0.33,0.53}{##1}}}
\expandafter\def\csname PYG@tok@mo\endcsname{\def\PYG@tc##1{\textcolor[rgb]{0.13,0.50,0.31}{##1}}}
\expandafter\def\csname PYG@tok@mi\endcsname{\def\PYG@tc##1{\textcolor[rgb]{0.13,0.50,0.31}{##1}}}
\expandafter\def\csname PYG@tok@kn\endcsname{\let\PYG@bf=\textbf\def\PYG@tc##1{\textcolor[rgb]{0.00,0.44,0.13}{##1}}}
\expandafter\def\csname PYG@tok@o\endcsname{\def\PYG@tc##1{\textcolor[rgb]{0.40,0.40,0.40}{##1}}}
\expandafter\def\csname PYG@tok@kr\endcsname{\let\PYG@bf=\textbf\def\PYG@tc##1{\textcolor[rgb]{0.00,0.44,0.13}{##1}}}
\expandafter\def\csname PYG@tok@s\endcsname{\def\PYG@tc##1{\textcolor[rgb]{0.25,0.44,0.63}{##1}}}
\expandafter\def\csname PYG@tok@kp\endcsname{\def\PYG@tc##1{\textcolor[rgb]{0.00,0.44,0.13}{##1}}}
\expandafter\def\csname PYG@tok@w\endcsname{\def\PYG@tc##1{\textcolor[rgb]{0.73,0.73,0.73}{##1}}}
\expandafter\def\csname PYG@tok@kt\endcsname{\def\PYG@tc##1{\textcolor[rgb]{0.56,0.13,0.00}{##1}}}
\expandafter\def\csname PYG@tok@sc\endcsname{\def\PYG@tc##1{\textcolor[rgb]{0.25,0.44,0.63}{##1}}}
\expandafter\def\csname PYG@tok@sb\endcsname{\def\PYG@tc##1{\textcolor[rgb]{0.25,0.44,0.63}{##1}}}
\expandafter\def\csname PYG@tok@k\endcsname{\let\PYG@bf=\textbf\def\PYG@tc##1{\textcolor[rgb]{0.00,0.44,0.13}{##1}}}
\expandafter\def\csname PYG@tok@se\endcsname{\let\PYG@bf=\textbf\def\PYG@tc##1{\textcolor[rgb]{0.25,0.44,0.63}{##1}}}
\expandafter\def\csname PYG@tok@sd\endcsname{\let\PYG@it=\textit\def\PYG@tc##1{\textcolor[rgb]{0.25,0.44,0.63}{##1}}}

\def\PYGZbs{\char`\\}
\def\PYGZus{\char`\_}
\def\PYGZob{\char`\{}
\def\PYGZcb{\char`\}}
\def\PYGZca{\char`\^}
\def\PYGZam{\char`\&}
\def\PYGZlt{\char`\<}
\def\PYGZgt{\char`\>}
\def\PYGZsh{\char`\#}
\def\PYGZpc{\char`\%}
\def\PYGZdl{\char`\$}
\def\PYGZhy{\char`\-}
\def\PYGZsq{\char`\'}
\def\PYGZdq{\char`\"}
\def\PYGZti{\char`\~}
% for compatibility with earlier versions
\def\PYGZat{@}
\def\PYGZlb{[}
\def\PYGZrb{]}
\makeatother

\renewcommand\PYGZsq{\textquotesingle}

\begin{document}

\maketitle
\tableofcontents
\phantomsection\label{index::doc}\begin{center}
Ce travail de maturité a été réalisé sous la direction de M. Donner
\end{center}

\setcounter{page}{1}

\chapter{Introduction}
\label{intro:introduction}\label{intro::doc}\label{intro:developpement-du-tableau-de-bord-professeur}
Ce document est la documentation du tableau de bord destiné à
être utilisé par les professeurs pour un site web d'e-learning pour les
mathématiques.

La problématique de ce travail s'intitule: «Développement du tableau de bord
professeur». Il a été nécessaire de faire attention à ne pas dépasser sur les
applications des différents contribuateurs à ce projet étant donné que le
tableau de bord sert de centre aux différentes applications du site.

Le dessein de cette documentation sera tout d'abord d'expliquer les
fonctionnalités de ce tableau de bord et de les illustrer. Cela permettra à un
professeur néophyte au tableau de bord de bien comprendre comment le prendre en
main.

De plus, la documentation expliquera le côté programmation de l'application.
En effet, cela comportera entre autre l'explication des différents fichiers, des
schémas résumant certaines caractéristiques de l'application ainsi que les
étapes à suivre pour démarrer un serveur Django depuis Cloud9.

La dernière partie consistera en une approche au développement dirigé par les
tests. Elle sera composée d'une explication des termes importants ainsi que de
l'explication du cycle fondamental dudit développement

Dans le projet final, qui consiste donc en un site web pour apprendre les
mathématiques, le tableau de bord sera le centre de toute activité. En effet,
c'est depuis cette application que les groupes seront créés, gérés et que les
exercices, quiz et cours seront gérés. Sans cette application, il ne serait
pas possible d'avoir une supervision des élèves ainsi qu'un cadre pour les aider
à s'améliorer.

Par exemple, pour aider différents élèves possédant les mêmes problèmes dans les
mêmes sujets, le professeur pourra créer un groupe dédié à ces étudiants et
pourra leur assigner des devoirs plus spécifiés pour les aider à progresser.
Cette application s'avérera donc être une grande aide pour les professeurs afin
qu'ils puissent mieux gérer les différents niveaux de leurs élèves.

Dans l'enseignement, il est plus difficile d'aider chaque élève selon leurs
difficultés et leurs aises. Cette application permettra donc un enseignement
plus précis pour que chaque élève puisse trouver ce qu'il lui plaîse.


\chapter{Fonctionnalités du tableau de bord}
\label{dashboard::doc}\label{dashboard:fonctionnalites-du-tableau-de-bord}
La première partie de cette documentation consiste en une explication des
différentes fonctionnalités du tableau de bord.

Il est important qu'un professeur puisse prendre correctement l'application en
main dès le début car cela lui permettra de l'utiliser à meilleur escient
et de ne pas passer à côté d'une fonctionnalité ou encore de perdre du temps
à comprendre son utilisation.


\section{Ajouter un groupe}
\label{dashboard:ajouter-un-groupe}
La fonctionnalité de base de ce tableau de bord est la création de groupe. En
créant un groupe, le professeur sera par la suite capable de le gérer en
supervisant les membres qui s'y trouvent mais aussi en y assignant des devoirs.

Pour créer un groupe, il suffit de se rendre sur Nouveau groupe, tout en bas
dans le menu de gauche. Cette action fera apparaître le formulaire de création
de groupe.
\begin{figure}[htbp]
\centering
\capstart

\includegraphics[width=0.500\linewidth]{Newclass.jpg}
\caption{Formulaire de création de groupe}\end{figure}

La seule exigence présente lors de la création d'un groupe est le nom. Une fois
le groupe créé, l'utilisateur actuel est automatiquement défini en tant que
professeur pour le groupe.

Le groupe précédemment créé sera désormais affiché en permanence dans le menu
de gauche du professeur, ce qui lui permet d'accéder à ses informations quand
il le désire.


\section{Gérer un groupe}
\label{dashboard:gerer-un-groupe}

\subsection{Gérer les membres du groupe}
\label{dashboard:gerer-les-membres-du-groupe}
Depuis cette page, le professeur peut gérer les membres qui sont actuellement
enregistrés dans le groupe.

Il peut tout d'abord rajouter les élèves ou professeurs qu'il souhaite en
entrant leur nom d'utilisateur dans le champ à disposition.
\begin{figure}[htbp]
\centering
\capstart

\includegraphics[width=0.500\linewidth]{class.jpg}
\caption{Page d'administration d'un groupe}\end{figure}

Si le nom d'utilisateur rentré correspond bien à un étudiant ou à un professeur,
cet utilisateur sera rajouté dans la liste des membres.
\begin{figure}[htbp]
\centering
\capstart

\includegraphics[width=0.500\linewidth]{classAjouterMembres.jpg}
\caption{Ce à quoi ressemble la page une fois que des membres ont été rajoutés}\end{figure}

Au contraire, si aucun utilisateur n'a été trouvé ou si l'utilisateur ne
correspond pas au rôle qui lui est donné (par exemple si c'est un
professeur et qu'il a été ajouté aux étudiants), un message d'erreur sera
retourné.
\begin{figure}[htbp]
\centering
\capstart

\includegraphics[width=0.500\linewidth]{classAjouterMembresEchec.jpg}
\caption{Message d'erreur retourné si l'utilisateur n'est pas valable}\end{figure}

Une fois ajouté, un membre peut facilement être retiré du groupe grâce au bouton
Retirer qui se trouve à côté de son nom.


\subsection{Gérer un devoir}
\label{dashboard:gerer-un-devoir}
Un professeur peut bien évidemment donner des devoirs à un groupe.

Un devoir peut être un exercice, un quiz ou un cours, et avoir été créé
par le professeur actuellement en ligne ou un autre.

Pour assigner un devoir, il suffit de savoir le numéro de l'exercice, quiz ou
cours, et de préciser grâce au menu à choix de quel type d'activité il s'agit.
\begin{figure}[htbp]
\centering
\capstart

\includegraphics[width=0.500\linewidth]{classDevoir.jpg}
\caption{Différents champs à compléter pour assigner un devoir}\end{figure}

Comme pour les fonctionnalités précédentes, si aucun exercice, quiz ou cours
n'a pu être associé au numéro entré, un message d'erreur sera renvoyé.
\begin{figure}[htbp]
\centering
\capstart

\includegraphics[width=0.500\linewidth]{devoirErreur.jpg}
\caption{Message d'erreur retourné si l'activité n'a pas pu être trouvée}\end{figure}

Un devoir peut être à tout moment retiré grâce au bouton Retirer à sa droite.


\section{Voir ses exercices}
\label{dashboard:voir-ses-exercices}
Dans le menu de gauche, il y a un bouton nommé Exercices. C'est depuis cette
page que le professeur pourra voir ses exercices, ses quiz et ses cours.
\begin{figure}[htbp]
\centering
\capstart

\includegraphics[width=0.500\linewidth]{exercices.jpg}
\caption{Ce à quoi ressemble la page Exercices}\end{figure}

Pour chaque activité que le professeur aura créée, il pourra voir le titre qu'il
lui a donné, la date à laquelle il l'a créée et son numéro qui lui sera utile
s'il veut l'assigner en tant que devoir à l'un de ses groupes.

Il peut bien évidemment supprimer une activité en utilisant le bouton Supprimer
se trouvant dans la dernière colonne du tableau.

Si le professeur souhaite créer une nouvelle activité, il n'a qu'à utiliser le
bouton Créer en haut du tableau qui le redirigera directement au formulaire de
création.


\section{Changer de mot de passe}
\label{dashboard:changer-de-mot-de-passe}
Peu importe sur quelle page il se trouve, le professeur peut accéder à un menu
déroulant en haut à droite de cette page.
\begin{figure}[htbp]
\centering
\capstart

\includegraphics[width=0.500\linewidth]{menuDeroulant.jpg}
\caption{Apparence du menu déroulant}\end{figure}

Dashboard amène le professeur sur l'accueil de son tableau de bord, Déconnexion le
déconnecte et Profil l'amène sur un formulaire de changement de mot de passe.

Pour le modifier, le professeur n'a qu'à remplir les deux champs et à valider.
Si tout a été rentré correctement, le mot de passe sera correctement modifié.
\begin{figure}[htbp]
\centering
\capstart

\includegraphics[width=0.500\linewidth]{passwordSuccess.jpg}
\caption{Message pour confirmer que le changement de mot de passe a correctement eu
lieu}\end{figure}

Au contraire, s'il y a une erreur, un message pour prévenir le
professeur sera retourné.
\begin{figure}[htbp]
\centering
\capstart

\includegraphics[width=0.500\linewidth]{passwordFail.jpg}
\caption{Message d'erreur retourné si les champs n'ont pas correctement été remplis}\end{figure}


\chapter{Guide du développeur}
\label{documentation:guide-du-developpeur}\label{documentation::doc}
La deuxième partie de cette documentation consiste en une explication du
fonctionnement de l'application du côté du développeur.

Il est donc expliqué comment les vues, modèles ou urls de cette application
fonctionnent et l'utilité de certains fichiers. Il y a aussi des schémas
permettant de mieux comprendre le fonctionnement du tableau de bord.


\section{Démarrer Django dans un workspace Cloud9}
\label{documentation:demarrer-django-dans-un-workspace-cloud9}
Cloud9 est un éditeur de code en ligne permettant de programmer depuis n'importe
où. Cette plateforme a été utilisé pour le développement de l'application.

Voici les quelques étapes à suivre pour démarrer Django dans Cloud9 \footnote{
«Cloud9 - Your development environment, in the cloud»,
consulté le 29.03.2015,
\href{https://c9.io/}{https://c9.io/}
}
\footnote{
«Configuration de Django 1.7 sous Cloud9»,
consulté le 24.03.2015,
\href{http://www.donner-online.ch/webtutos/django/c9config.html}{http://www.donner-online.ch/webtutos/django/c9config.html}
}.

Il faut tout d'abord créer un workspace.

\textbf{Attention, si on souhaite utiliser Python3 il est conseillé de créer un
workspace de type Custom, comme sur la photo suivante:}
\begin{figure}[htbp]
\centering
\capstart

\includegraphics[width=0.600\linewidth]{Workspace.jpg}
\caption{Créer un workspace de type Custom}\end{figure}

Une fois sur le dépot, il faut tout d'abord installer Django avec
la commande suivante:

\begin{Verbatim}[commandchars=\\\{\}]
sudo pip3 install django==1.7
\end{Verbatim}

Django installé, il faut démarrer un projet avec la commande suivante:

\begin{Verbatim}[commandchars=\\\{\}]
django\PYGZhy{}admin startproject nom\PYGZhy{}pour\PYGZhy{}le\PYGZhy{}projet
\end{Verbatim}

Maintenant que le projet existe, il est possible de créer autant d'applications
que souhaité avec la commande:

\begin{Verbatim}[commandchars=\\\{\}]
python3 manage.py startapp nom\PYGZhy{}pour\PYGZhy{}l\PYGZsq{}application
\end{Verbatim}

Pour lancer le serveur, il est nécessaire de taper cette commande:

\begin{Verbatim}[commandchars=\\\{\}]
python3 manage.py runserver \PYGZdl{}IP:\PYGZdl{}PORT
\end{Verbatim}


\section{Navigation}
\label{documentation:navigation}\begin{figure}[htbp]
\centering
\capstart

\includegraphics[width=0.600\linewidth]{navigation.jpg}
\caption{Schéma de navigation du site}\end{figure}

Ce schéma explique les relations qui existent entre les différentes pages. Plus
précisément, comment accéder une page depuis une autre.

Il est important de noter que le menu déroulant ainsi que les pages Exercices,
Nouveau groupe et la page d'une classe peuvent être atteintes depuis n'importe
quelle page du tableau de bord.


\section{Use Cases}
\label{documentation:use-cases}\begin{figure}[htbp]
\centering
\capstart

\includegraphics[width=0.600\linewidth]{UseCases.jpg}
\caption{Schéma résumant les actions qui se déroulent selon les utilisations du
professeur}\end{figure}

Ce schéma explique les différentes actions qui se passent lorsque le professeur
veut utiliser une des fonctionnalités du tableau de bord.

Par exemple, s'il veut ajouter un élève dans un de ses groupes, il n'a qu'à
entrer son nom, le serveur ira le chercher et l'ajoutera dans le groupe.


\section{Dossier \texttt{static}}
\label{documentation:dossier-static}
Le dossier static est utilisé pour garder tous les fichiers tel que
les fichiers CSS ou les fichiers Javascript.
Dans cette application, il contient les dossiers suivants:
\begin{itemize}
\item {} 
\code{bower\_components}: ce dossier contient tous les éléments du front-end
qui possèdent des dépendances, comme les fichiers \code{bootstrap} ou des
fichiers de base pour \code{jquery}. Le dossier \code{bower\_components} contient
les fichiers relatifs au thème Bootstrap utilisé. Plus précisément, il
contient le \code{css} et le \code{javascript}.

\item {} 
\code{css}: dans ce dossier se trouvent tous les fichiers \code{css} qui sont
nécessaires pour le design du site. La différence entre les fichiers qui
se trouvent dans ce dossier et les fichiers \code{css} du dossier
\code{bower\_components} est que les premiers servent de base et ne sont pas
adaptatifs alors que les derniers permettent les changements de place et
de taille que nous offre l'adaptivité de Bootstrap.

\item {} 
\code{fonts}: le dossier \code{fonts} de l'application tableau de bord contient toutes
les informations relatives aux petits signes (\code{glyphicons}) qui sont
utilisés dans le tableau de bord, comme le + devant «Nouveau groupe».

\end{itemize}
\begin{figure}[htbp]
\centering
\capstart

\includegraphics[width=0.600\linewidth]{class.jpg}
\caption{Exemples de \code{glyphicons} dans le menu de gauche: l'oeil ou la maison}\end{figure}


\section{Gabarits}
\label{documentation:gabarits}
Dans le dossier \code{templates} se trouvent tous les fichiers \code{html} servant
de gabarits à l'application:
\begin{itemize}
\item {} 
Le gabarit \code{classe.html} contient le gabarit utilisé pour l'affichage
des groupes. Il affiche tout d'abord le nom du groupe ainsi que la date de
sa création, puis chaque élève et professeur ainsi qu'un bouton pour
les retirer du groupe. Dans les tableaux affichant les élèves et professeur
est aussi affiché un bouton pour rajouter des membres. S'il est impossible
d'ajouter le membre dont le nom a été rentré, le gabarit retourne un
message d'erreur.

Il affiche enfin les devoirs par type d'activité (exercice, quiz, cours) et
un bouton pour les retirer. Il y a aussi un bouton pour assigner des
devoirs. De nouveau, si aucune activité n'a été trouvée, le gabarit
retourne un message d'erreur.

\item {} 
Le gabarit \code{newclass.html}, lui, sert à créer un nouveau groupe qui pourra
ensuite être supervisé. Il ne fait qu'afficher un champ pour le nom et un
bouton de confirmation. Une fois le groupe créé, un message de confirmation
est retourné.

\item {} 
Le gabarit \code{index.html} ne contient pour le moment que le nom
d'utilisateur du professeur actuellement connecté. Il contiendra plus tard
des statistiques quant aux groupes ou aux activités du professeur.

\item {} 
Le gabarit \code{exercises.html} affiche les exercices, les quiz et les cours
qui ont été créés par le professeur. Ces activités sont supprimables depuis
cette page et le professeur a la possibilité d'accéder aux formulaires de
création d'activité. Il peut aussi voir quand ces activités ont été créées.

\item {} 
Le gabarit \code{profile.html} sert au changement de mot de passe. Il affiche
deux champ qui doivent être remplis de façon identique. Si le changement
de mot de passe a bien pu avoir lieu, un message de confirmation est
retourné. Dans le cas contraire, un message d'erreur s'affiche.

\end{itemize}

Ces 5 gabarits ont tous la même structure de base:
\begin{itemize}
\item {} 
Une bande au sommet de la page qui, une fois les applications ensemble,
amènera l'utilisateur à ces applications.

Il y a aussi un menu déroulant permettant d'accéder au tableau de bord,
au changement de mot de passe et permettant aussi à l'utilisateur de se
déconnecter.

\item {} 
Un menu à gauche de la page permettant d'accéder aux exercices, aux
différents groupes qui sont tous ajoutés en liste et à l'option de création
de groupe.

\end{itemize}


\section{Fichiers importants}
\label{documentation:fichiers-importants}
Les applications Django possèdent les fichiers de base suivants:
\begin{itemize}
\item {} 
\code{models.py} qui est utilisé pour créer les différents modèles et leur
attribuer des champs.

\item {} 
\code{admin.py} est utilisé pour signaler à Django quels sont les modèles qui
doivent apparaître dans l'application admin. Une fois qu'ils y apparaissent,
il est possible de créer, modifier ou supprimer n'importe quel objet depuis
cette application.

\item {} 
Le fichier \code{forms.py} est celui dans lequel on peut entrer les différents
formulaires dont l'on a besoin pour l'application.

\item {} 
C'est dans \code{views.py} que l'on peut stocker des variables nécessaires
dans certains gabarits, mais aussi réaliser certaines actions comme la
suppression d'un objet. A la fin d'une vue, on retourne souvent un fichiers
\code{html} ou on redirige vers une autre vue.

\item {} 
Le fichier \code{urls.py} contient les informations concernant les différentes
urls accessibles par l'utilisateur et quelles vues sont censées être
utilisées.

\end{itemize}


\subsection{Fichiers uniques de Django}
\label{documentation:fichiers-uniques-de-django}
On peut modifier le fichier \code{settings.py} afin de définir la zone temporelle
dans laquelle on se trouve, mais aussi les applications qu'un projet doit
gérer ou encore l'emplacement du fichier \code{static}. Il sert donc de
configuration de base pour un projet.

Il y a aussi un autre fichiers \code{urls.py} qui, lui, est très utile si l'on doit
s'occuper de plusieurs applications à la fois. En effet, on peut définir le
début de l'url et rediriger vers un autre fichier \code{urls.py}.


\section{Modèles}
\label{documentation:modeles}

\subsection{Modèles utilisés pour le tableau de bord}
\label{documentation:modeles-utilises-pour-le-tableau-de-bord}
Il y a tout d'abord le modèle \code{BaseProfile} qui découle de \code{User} et qui,
comme son nom l'indique, va servir de profil de base pour le modèle \code{Teacher}
et \code{Student}.

L'utilisateur Django possède de base les caractéristiques suivantes \footnote{
«django.contrib.auth»,
consulté le 23.03.2015,
\href{https://docs.djangoproject.com/en/1.7/ref/contrib/auth/}{https://docs.djangoproject.com/en/1.7/ref/contrib/auth/}
}:
\begin{itemize}
\item {} 
\code{username}: nom d'utilisateur

\item {} 
\code{first\_name}: prénom

\item {} 
\code{last\_name}: nom

\item {} 
\code{email}: adresse courriel

\item {} 
\code{password}: mot de passe

\item {} 
\code{group}: les relations avec le modèle \code{Group} de Django

\item {} 
\code{user\_permissions}: les relations avec le modèle \code{Permission} de Django

\item {} 
\code{is\_staff}: si l'utilisateur peut accéder à l'application admin

\item {} 
\code{is\_active}: définit si l'utilisateur doit être considéré comme actif ou
non

\item {} 
\code{is\_superuser}: définit si l'utilisateur a tous les droits

\item {} 
\code{last\_login}: dernière connexion de l'utilisateur

\item {} 
\code{date\_joined}: date de création de l'utilisateur

\end{itemize}

Car un professeur a besoin de voir ses exercices, quiz et cours, et pourra les
assigner en tant que devoirs à un groupe, les modèles Exercise, Quiz et Course
ont tous les trois été apportés.

Il y a ensuite le modèle \code{Group}, qui n'est pas le même que celui implémenté
de base avec Django, car c'est celui qui a été utilisé pour les groupes d'un
professeur. Les membres sont ajoutés par le biais du modèle \code{Groupmembers}
qui sert de table intermédiaire entre \code{Student} ainsi que \code{Teacher} et
\code{Group}. Le modèle \code{AssignHomework}, qui est aussi une table intermédiaire,
sert à l'affectation de devoirs entre \code{Exercise}, \code{Quiz}, \code{Course} et
\code{Group}.


\subsection{Diagramme UML}
\label{documentation:diagramme-uml}\begin{figure}[htbp]
\centering
\capstart

\includegraphics[width=0.600\linewidth]{UML.jpg}
\caption{Schéma résumant les relations entre les différents modèles}\end{figure}

Sur ce schéma, les types de lien existant entre les différents modèles sont plus
visibles:
\begin{itemize}
\item {} 
\code{0..*} signifie qu'un objet peut possèder entre zéro et l'infini objets
appartenant à l'autre modèle

\item {} 
\code{1} signifie qu'il ne peut en posséder qu'un seul

\end{itemize}

Il est aussi facile à voir la place que prennent les tables intermédiaires
\code{AssignHomework} et \code{GroupMembers}: elles servent de ponts entre deux
modèles.


\section{Vues}
\label{documentation:vues}
Toutes les vues vont devoir chercher le professeur correspondant à
l'utilisateur actuellement connecté. Cela permettra à chaque fois d'aller
chercher les données correspondantes comme le nom d'utilisateur toujours
affiché sur le menu déroulant en haut à droite de la page.

La vue \code{home} sert uniquement à distinguer la première lettre du nom
d'utilisateur pour qu'apparaisse dans le gabarit «de» ou «d'».

La vue \code{exercises}, elle, ne cherche rien de plus. L'utilisateur nous
permettra d'accéder aux exercices, quiz et cours qui lui sont associés mais
tout ceci est directement recherché dans le gabarit.

C'est grâce à la vue \code{newgroup} qu'un professeur peut créer un groupe. S'il
veut créer un groupe, la vue se contentera de créer un groupe associé au nom
et de créer un lien entre le professeur et le groupe grâce à la table
intermédiaire \code{GroupMembers}. La variable \code{success} a pour utilité
d'afficher un message de confirmation dans le gabarit \code{newclass.html} une
fois le groupe correctement créé.

La vue \code{profil} est celle utilisée pour le changement de mot de passe. Elle
compare les deux mots de passe entrés. Si les deux mots de passe correspondent,
le mot de passe est attribué à l'utilisateur et, grâce au gabarit
\code{profile.html} et à la variable \code{success}, un message est retourné pour
confirmer le changement. Dans le cas contraire, un message d'erreur est
retourné.

La vue \code{groupe}, elle, est composée de plusieurs actions qui dépendent de la
forme qui a été remplie.
\begin{itemize}
\item {} 
Il y a tout d'abord \code{addTeacher} qui, quand l'utilisateur entre le nom
d'utilisateur d'un autre professeur pour l'ajouter dans un groupe existant,
va créer un object \code{GroupMembers} entre ce professeur et le groupe actuel
pour qu'il fasse parti de ce groupe. Il se passe la même chose pour
\code{addStudent} si le même utilisateur décide d'ajouter un élève.

\item {} 
Pour assigner un devoir à un groupe, la vue va utiliser \code{assignHomework}
qui, selon le genre d'activité et le numéro qui ont été sélectionnés par
l'utilisateur, va chercher l'activité et créer un objet \code{AssignHomework}
qui va lier l'exercice, le quiz ou le cours au groupe.

\item {} 
Pour supprimer un groupe, il y a d'abord l'utilisation de \code{deleteClass}
qui va uniquement servir à l'apparition d'un deuxième bouton qui activera
\code{deleteClassConfirm}, qui supprimera le groupe et donc tous les objets
\code{AssignHomework} et \code{GroupMembers} avec lesquels il était associé.

\end{itemize}

Cette vue va par la suite retourner le gabarit \code{classe.html} avec les
variables définies au début qui apparaîtront sur la page.

Finalement, quelques vues ont été réalisées pour des actions plus complexes.
Par exemple, \code{deleteFromGroup} avait besoin de deux variables, \code{member\_id}
et \code{group\_id}. Cette vue a donc été liée à une url nécessitant ces deux
variables. La vue \code{deleteFromGroup}, composée de \code{deleteStudent} et
\code{deleteTeacher}, servent à retirer les membres d'un groupe en supprimant
l'objet \code{GroupMembers} qui les liait. La vue \code{deleteActivity}, qui elle est
composée de \code{deleteExercise}, \code{deleteQuiz} et \code{deleteCourse} sert à
supprimer une activité depuis son tableau de bord. Enfin, \code{deleteHomework} permet
au professeur de retirer un devoir précédemment assigné selon le type d'activité
auquel il correspond.


\section{Urls}
\label{documentation:urls}
Les urls \code{home}, \code{group\_view}, \code{exercises}, \code{newgroup} et \code{profil}
redirigent simplement aux vues du même nom.

Les urls \code{deleteFromGroup}, \code{deleteActivity} et \code{deleteHomework}, elles,
sont reliées aux vues du même nom qui permettent certaines actions dépendantes
de variables très précises. Pour réaliser ceci, des formes ont été créées dans
les gabarits redirigeant à ces urls et possédant les variables nécessaires afin
que le programme puisse aller chercher les objets souhaités et permettre, par
exemple, la suppression d'une activité.
\paragraph{Note de bas de page}


\chapter{Développement dirigé par les tests}
\label{tdd::doc}\label{tdd:developpement-dirige-par-les-tests}
Le développement dirigé par les tests, grossièrement traduit de l'anglais Test
Driven Development, est une technique de développement utilisée par beaucoup de
programmeurs.

En effet, c'est grâce à cette technique que l'on peut le mieux s'assurer de la
fonctionnalité du site et de la simplicité du code. Dans un travail de longue
haleine, cette méthode devient nécessaire pour ne pas être redondant dans son
code et pour le rendre le plus clair possible.

Dans cette partie, nous allons nous intéresser au côté théorique de ce
type de développement.

Le développement dirigé par les tests est composé de deux tests importants et
bien différents, même s'il existe un troisième qui est utilisé dans un
context très précis.


\section{Les tests fonctionnels}
\label{tdd:les-tests-fonctionnels}
Le test fonctionnel, comme son nom l'indique, cherche à tester la fonctionnalité
du site. Plus précisément, il se met du côté de l'utilisateur du site. Il va par
exemple vérifier que les titres et les textes apparaissent, mais il va aussi
regarder si les différents boutons ou champs de textes fonctionnent \footnote{
PERCIVAL, Harry J.W., «Test Driven Development With Python», publié
le 19 juin 2014
}.

En général, on remplit ces tests de commentaires qui raccontent une histoire
pour pouvoir s'y retrouver plus facilement quand on le lit. Par exemple,
on pourrait racconter l'histoire d'un professeur qui découvre un site web
d'e-learning pour les mathématiques et qui d'abord crée un compte, puis
essaie de créer une classe.


\section{Les tests unitaires}
\label{tdd:les-tests-unitaires}
Le test unitaire, lui, se base plus sur le point de vue du programmeur. Si on
pouvait considérer le test fonctionnel comme un test externe, le test unitaire,
lui, serait le test interne. Ce qu'il teste ne sera jamais vu, car il permet de
vérifier que le code fonctionne comme prévu \footnotemark[1].


\section{Les tests d'approbation}
\label{tdd:les-tests-d-approbation}
Plus rare, il existe aussi le test d'approbation qui peut même amener à un
développement plus précis: Acceptance Test Driven Development (ATDD), ou
développement dirigé par les tests d'approbation \footnote{
«Acceptance Test Driven Development (ATDD): An Overview»,
consulté le 25.03.2015,
\href{http://testobsessed.com/2008/12/acceptance-test-driven-development-atdd-an}{http://testobsessed.com/2008/12/acceptance-test-driven-development-atdd-an}-
overview/
}.

Le grand avantage de ces tests est qu'ils permettent un meilleur travail de
groupe. En effet, l'équipe travaillant sur le projet se réunit et décide des
critères de base de ce projet qui seront ensuite écrit en test. Par conséquent,
toute l'équipe est d'accord sur ces critères et il sera ensuite facile de voir
s'ils sont respectés.


\section{Le cycle du développement dirigé par les tests}
\label{tdd:le-cycle-du-developpement-dirige-par-les-tests}
Quand on veut programmer à l'aide du développement dirigé par les tests, on
tente de suivre un certain cycle:
\begin{enumerate}
\item {} 
Tout d'abord, il est \textbf{impératif} d'écrire un test avant même d'écrire
n'importe quelle ligne de code. En effet, l'idée du développement dirigé par
les tests est d'être sûr qu'un test échoue avant d'écrire notre code. Le
test peut être fonctionnel, unitaire, ou d'approbation (dans le cadre du
ATDD), cela dépendant de la partie de l'application que l'on souhaite
développer. Le test va évidemment retourner un message négatif, mais c'est
normal étant donné que rien n'a été codé concernant la fonctionnalité
testée.

\item {} 
Ensuite seulement, le but est d'écrire un minimum de code possible
pour que le test précédemment lancé fonctionne. Il faut donc réussir
à ce que le test, une fois relancé, retourne un résultat positif. Il faut
faire attention à ne pas développer une fonctionnalité qui n'est pas dans
le test

\item {} 
Une fois le code écrit, on peut relancer ce test. Si le résultat
est positif, on peut passer à l'étape suivante. Dans le cas contraire,
il faudra refaire les étapes 2 et 3 jusqu'à ce que le test fonctionne.

\item {} 
Finalement, il ne reste qu'à restructurer le code précédemment écrit pour
qu'il soit plus lisible. Il faut faire très attention durant cette étape à
ne rien ajouter ou enlever. Le code doit garder le même résultat.

\end{enumerate}

Une fois que ces 4 étapes ont été effectuées, il ne reste qu'à
recommencer avec une autre fonctionnalité de l'application, jusqu'à
que celle-ci soit finie.
\begin{figure}[htbp]
\centering
\capstart

\includegraphics[width=0.700\linewidth]{TDD.png}
\caption{Schéma résumant les 4 étapes du développement dirigé par les tests}\end{figure}


\section{Gain de temps?}
\label{tdd:gain-de-temps}
En lisant ces 4 étapes répétitives, on ne peut que se demander si le Test
Driven Development et son cycle compliqué est réellement un atout et un gain
de temps pour le programmeur.

Il est clair que, sur un travail de petite taille, tout tester n'aurait pas
énormément de sens, car tout peut être facilement essayable par soi-même.
Dans le cas d'un travail d'une certaine consistance, ce n'est pas pareil.
C'est uniquement en testant que l'on peut être sûr de son code, car cela
signifie qu'il est valide et devrait le rester.
\paragraph{Note de bas de page}


\chapter{Conclusion}
\label{conclu::doc}\label{conclu:conclusion}
En conclusion, si l'on pose un regard critique sur le tableau de bord destiné
aux professeurs présenté dans cette documentation, nous pouvons voir qu'il
possède des limites.

En effet, bien que la création de groupe soit un outil très utile, la faculté
de voir les statistiques de ses élèves ou encore les résultats aux différentes
acitivités serait un bien incommensurable pour que le professeur puisse
s'adapter aux besoin des étudiants et puisse leur proposer des exercices
adaptés. De plus, il aurait été utile que le professeur puisse observer les
résultats aux exercices de certains élèves.

A l'avenir, il serait envisageable de pouvoir accéder à un tableau mettant
en valeur les résultats obtenus par les élèves à des activités précises,
permettant de comparer et d'aider les élèves en difficulté avant un examen
ou encore de les ajouter à un groupe d'étudiants aux même difficultés.


\chapter{Sources}
\label{source:sources}\label{source::doc}

\section{Bibliographie}
\label{source:bibliographie}
ALEXIS, Pierre et BERSINI, Hugues, «Apprendre la programmation web avec Python
et Django», chapitres 5 et 10, publié le 9 novembre 2012

PERCIVAL, Harry J.W., «Test Driven Development With Python», chapitres 1 à 9,
publié le 19 juin 2014


\section{Webographie}
\label{source:webographie}
«Dash \textbar{} Learn HTML, CSS, JavaScript with our free online tutorial \textbar{} General
Assembly»,
consulté le 16.08.2014,
\href{https://dash.generalassemb.ly/}{https://dash.generalassemb.ly/}

«France-IOI - Cours et problèmes»,
consulté le 29.08.2014,
\href{http://www.france-ioi.org/algo/chapters.php}{http://www.france-ioi.org/algo/chapters.php}

«django.contrib.auth»,
consulté le 23.03.2015,
\href{https://docs.djangoproject.com/en/1.7/ref/contrib/auth/}{https://docs.djangoproject.com/en/1.7/ref/contrib/auth/}

«Configuration de Django 1.7 sous Cloud9»,
consulté le 24.03.2015,
\href{http://www.donner-online.ch/webtutos/django/c9config.html}{http://www.donner-online.ch/webtutos/django/c9config.html}

«Acceptance Test Driven Development (ATDD): An Overview»,
consulté le 25.03.2015,
\href{http://testobsessed.com/2008/12/acceptance-test-driven-development-atdd-an-overview/}{http://testobsessed.com/2008/12/acceptance-test-driven-development-atdd-an-overview/}

«Cloud9 - Your development environment, in the cloud»,
consulté le 29.03.2015,
\href{https://c9.io/}{https://c9.io/}


\chapter{Annexes}
\label{annexes::doc}\label{annexes:annexes}

\section{Déclaration personnelle}
\label{annexes:declaration-personnelle}
Nom: Oberson

Prénom: Bryan

Adresse: Route du Lac Lussy 3

1. Je certifie que le travail
\textbf{Développement du tableau de bord professeur}
a été réalisé par moi conformément au Guide de travail des collèges et aux
Lignes directrices de la DICS concernant la réalisation du Travail de maturité.

2. Je prends connaissance que mon travail sera soumis à une vérification de la
mention correcte et complète de ses sources, au moyen d’un logiciel de détection
de plagiat. Pour assurer ma protection, ce logiciel sera également utilisé pour
comparer mon travail avec des travaux écrits remis ultérieurement, afin d’éviter
des copies et de protéger mon droit d’auteur. En cas de soupçon d’atteintes à
mon droit d’auteur, je donne mon accord à la direction de l’école pour
l’utilisation de mon travail comme moyen de preuve.
\begin{enumerate}
\setcounter{enumi}{2}
\item {} 
Je m'engage à ne pas rendre public mon travail avant l'évaluation finale.

\end{enumerate}

4. Je m’engage à respecter la Procédure d’archivage des travaux de
maturité/travaux personnels/travaux de maturité spécialisée en vigueur dans mon
école.

5. J’autorise la consultation de mon travail par des tierces personnes à des
fins pédagogiques et/ou d’information interne à l’école :
\begin{itemize}
\item {} 
oui

\item {} 
non (car il contient des données personnelles et sensibles.)

\end{itemize}

Lieu, date : \_\_\_\_\_\_\_\_\_\_\_\_\_\_\_\_\_\_\_\_\_\_\_\_\_\_\_\_\_\_\_\_\_\_\_\_\_\_\_\_\_\_\_\_\_\_\_\_\_\_\_\_\_\_

Signature : \_\_\_\_\_\_\_\_\_\_\_\_\_\_\_\_\_\_\_\_\_\_\_\_\_\_\_\_\_\_\_\_\_\_\_\_\_\_\_\_\_\_\_\_\_\_\_\_\_\_\_\_\_\_


\section{Remerciements}
\label{annexes:remerciements}
Je tiens à remercier M. Cédric Donner, superviseur du séminaire, pour sa
disponibilité à tout moment, ses conseils quant au travail ainsi que son
encouragement tout au long de ce travail.

Je remercie Keran Kocher (3GY3), Benoît Léo Maillard (3GY1), Florian
Genilloud (3GY2), Daniel Nunes Silva (3GY1) et Sébastien Chuat (3GY3), qui ont
aussi travaillé sur ce projet, pour leurs réponses à mes questions et leur
entraide dans les sujets difficiles.

Je remercie aussi Léa Gobet (3GY3) pour m'avoir aidé quant à la mise en forme
de mon travail.

Merci encore à Alexandre Currat (3GY7) pour avoir essayé l'application et
m'avoir prévenu de tous les bogues présents ainsi que les améliorations
envisageables.


\section{Code source}
\label{annexes:code-source}

\subsection{\texttt{models.py}}
\label{annexes:models-py}
\begin{Verbatim}[commandchars=\\\{\},numbers=left,firstnumber=1,stepnumber=1]
\PYG{k+kn}{from} \PYG{n+nn}{django.db} \PYG{k+kn}{import} \PYG{n}{models}
\PYG{k+kn}{from} \PYG{n+nn}{django.contrib.auth.models} \PYG{k+kn}{import} \PYG{n}{User}


\PYG{c}{\PYGZsh{}Profile de base découlant de User}

\PYG{k}{class} \PYG{n+nc}{BaseProfile}\PYG{p}{(}\PYG{n}{models}\PYG{o}{.}\PYG{n}{Model}\PYG{p}{)}\PYG{p}{:}
    \PYG{n}{user} \PYG{o}{=} \PYG{n}{models}\PYG{o}{.}\PYG{n}{OneToOneField}\PYG{p}{(}\PYG{n}{User}\PYG{p}{)} \PYG{c}{\PYGZsh{}Donne les attributs de User à BaseProfile}
    \PYG{n}{avatar} \PYG{o}{=} \PYG{n}{models}\PYG{o}{.}\PYG{n}{ImageField}\PYG{p}{(}\PYG{n}{null}\PYG{o}{=}\PYG{n+nb+bp}{True}\PYG{p}{,} \PYG{n}{blank}\PYG{o}{=}\PYG{n+nb+bp}{True}\PYG{p}{,} \PYG{n}{upload\PYGZus{}to}\PYG{o}{=}\PYG{l+s}{\PYGZdq{}}\PYG{l+s}{avatars/}\PYG{l+s}{\PYGZdq{}}\PYG{p}{)}
        
    \PYG{k}{class} \PYG{n+nc}{Meta}\PYG{p}{:}
        \PYG{n}{abstract} \PYG{o}{=} \PYG{n+nb+bp}{True}


\PYG{c}{\PYGZsh{}Les deux modèles héritent de BaseProfile et donc de User}

\PYG{k}{class} \PYG{n+nc}{Teacher}\PYG{p}{(}\PYG{n}{BaseProfile}\PYG{p}{)}\PYG{p}{:}

    \PYG{k}{def} \PYG{n+nf}{\PYGZus{}\PYGZus{}str\PYGZus{}\PYGZus{}}\PYG{p}{(}\PYG{n+nb+bp}{self}\PYG{p}{)}\PYG{p}{:}
        \PYG{k}{return} \PYG{l+s}{\PYGZdq{}}\PYG{l+s}{Professeur \PYGZob{}0\PYGZcb{}}\PYG{l+s}{\PYGZdq{}}\PYG{o}{.}\PYG{n}{format}\PYG{p}{(}\PYG{n+nb+bp}{self}\PYG{o}{.}\PYG{n}{user}\PYG{o}{.}\PYG{n}{username}\PYG{p}{)}

\PYG{k}{class} \PYG{n+nc}{Student}\PYG{p}{(}\PYG{n}{BaseProfile}\PYG{p}{)}\PYG{p}{:}

    \PYG{k}{def} \PYG{n+nf}{\PYGZus{}\PYGZus{}str\PYGZus{}\PYGZus{}}\PYG{p}{(}\PYG{n+nb+bp}{self}\PYG{p}{)}\PYG{p}{:}
        \PYG{k}{return} \PYG{l+s}{\PYGZdq{}}\PYG{l+s}{Etudiant \PYGZob{}0\PYGZcb{}}\PYG{l+s}{\PYGZdq{}}\PYG{o}{.}\PYG{n}{format}\PYG{p}{(}\PYG{n+nb+bp}{self}\PYG{o}{.}\PYG{n}{user}\PYG{o}{.}\PYG{n}{username}\PYG{p}{)}
        

\PYG{c}{\PYGZsh{}}
\PYG{c}{\PYGZsh{} Modèle de Keran pour les cours}
\PYG{c}{\PYGZsh{}}

\PYG{k}{class} \PYG{n+nc}{Course}\PYG{p}{(}\PYG{n}{models}\PYG{o}{.}\PYG{n}{Model}\PYG{p}{)}\PYG{p}{:}
    \PYG{n}{title} \PYG{o}{=} \PYG{n}{models}\PYG{o}{.}\PYG{n}{CharField}\PYG{p}{(}\PYG{n}{max\PYGZus{}length}\PYG{o}{=}\PYG{l+m+mi}{30}\PYG{p}{,} \PYG{n}{unique}\PYG{o}{=}\PYG{n+nb+bp}{True}\PYG{p}{)}
    \PYG{n}{description} \PYG{o}{=} \PYG{n}{models}\PYG{o}{.}\PYG{n}{TextField}\PYG{p}{(}\PYG{p}{)}
    \PYG{n}{difficulty} \PYG{o}{=} \PYG{n}{models}\PYG{o}{.}\PYG{n}{IntegerField}\PYG{p}{(}\PYG{p}{)}
    \PYG{n}{published} \PYG{o}{=} \PYG{n}{models}\PYG{o}{.}\PYG{n}{BooleanField}\PYG{p}{(}\PYG{n}{default}\PYG{o}{=}\PYG{n+nb+bp}{False}\PYG{p}{)}
    
    \PYG{n}{author} \PYG{o}{=} \PYG{n}{models}\PYG{o}{.}\PYG{n}{ForeignKey}\PYG{p}{(}\PYG{n}{Teacher}\PYG{p}{)}
    \PYG{c}{\PYGZsh{}chapter = models.ForeignKey(\PYGZsq{}teachers.Chapter\PYGZsq{}, related\PYGZus{}name=\PYGZdq{}courses\PYGZdq{})}
    \PYG{n}{favorites} \PYG{o}{=} \PYG{n}{models}\PYG{o}{.}\PYG{n}{ManyToManyField}\PYG{p}{(}\PYG{n}{User}\PYG{p}{,} \PYG{n}{related\PYGZus{}name}\PYG{o}{=}\PYG{l+s}{\PYGZdq{}}\PYG{l+s}{favorite\PYGZus{}courses}\PYG{l+s}{\PYGZdq{}}\PYG{p}{,} \PYG{n}{blank}\PYG{o}{=}\PYG{n+nb+bp}{True}\PYG{p}{,} \PYG{n}{null}\PYG{o}{=}\PYG{n+nb+bp}{True}\PYG{p}{)}
    \PYG{c}{\PYGZsh{} videos = models.ManyToManyField(Video)}
    \PYG{c}{\PYGZsh{} images = models.ManyToManyField(Image)}
    \PYG{c}{\PYGZsh{} definitions = models.ManyToManyField(Definition)}
    
    \PYG{n}{created\PYGZus{}at} \PYG{o}{=} \PYG{n}{models}\PYG{o}{.}\PYG{n}{DateTimeField}\PYG{p}{(}\PYG{n}{auto\PYGZus{}now\PYGZus{}add}\PYG{o}{=}\PYG{n+nb+bp}{True}\PYG{p}{)}
    \PYG{n}{updated\PYGZus{}at} \PYG{o}{=} \PYG{n}{models}\PYG{o}{.}\PYG{n}{DateTimeField}\PYG{p}{(}\PYG{n}{auto\PYGZus{}now}\PYG{o}{=}\PYG{n+nb+bp}{True}\PYG{p}{)}

    \PYG{k}{def} \PYG{n+nf}{\PYGZus{}\PYGZus{}str\PYGZus{}\PYGZus{}}\PYG{p}{(}\PYG{n+nb+bp}{self}\PYG{p}{)}\PYG{p}{:}
      \PYG{k}{return} \PYG{n+nb+bp}{self}\PYG{o}{.}\PYG{n}{title}
      
\PYG{c}{\PYGZsh{}}
\PYG{c}{\PYGZsh{} Modèle de Florian pour les exercices}
\PYG{c}{\PYGZsh{}}

\PYG{k}{class} \PYG{n+nc}{Exercise}\PYG{p}{(}\PYG{n}{models}\PYG{o}{.}\PYG{n}{Model}\PYG{p}{)}\PYG{p}{:}
    
    \PYG{n}{owner} \PYG{o}{=} \PYG{n}{models}\PYG{o}{.}\PYG{n}{ForeignKey}\PYG{p}{(}\PYG{n}{Teacher}\PYG{p}{)}  \PYG{c}{\PYGZsh{} créateur de l\PYGZsq{}exercice   }
    \PYG{n}{created\PYGZus{}on} \PYG{o}{=} \PYG{n}{models}\PYG{o}{.}\PYG{n}{DateTimeField}\PYG{p}{(}\PYG{n}{auto\PYGZus{}now\PYGZus{}add}\PYG{o}{=}\PYG{n+nb+bp}{True}\PYG{p}{)} \PYG{c}{\PYGZsh{} Date de création}
    \PYG{n}{updated\PYGZus{}on} \PYG{o}{=} \PYG{n}{models}\PYG{o}{.}\PYG{n}{DateTimeField}\PYG{p}{(}\PYG{n}{auto\PYGZus{}now}\PYG{o}{=}\PYG{n+nb+bp}{True}\PYG{p}{)}
    \PYG{n}{title} \PYG{o}{=} \PYG{n}{models}\PYG{o}{.}\PYG{n}{CharField}\PYG{p}{(}\PYG{n}{max\PYGZus{}length}\PYG{o}{=}\PYG{l+m+mi}{30}\PYG{p}{)} \PYG{c}{\PYGZsh{} C\PYGZsq{}est le titre de l\PYGZsq{}exercice}
    \PYG{n}{equation} \PYG{o}{=} \PYG{n}{models}\PYG{o}{.}\PYG{n}{CharField}\PYG{p}{(}\PYG{n}{max\PYGZus{}length}\PYG{o}{=}\PYG{l+m+mi}{50}\PYG{p}{)} \PYG{c}{\PYGZsh{} C\PYGZsq{}est l\PYGZsq{}équation entrée par le professeur}
    \PYG{n}{grade} \PYG{o}{=} \PYG{n}{models}\PYG{o}{.}\PYG{n}{CharField}\PYG{p}{(}\PYG{n}{max\PYGZus{}length}\PYG{o}{=}\PYG{l+m+mi}{60}\PYG{p}{)} \PYG{c}{\PYGZsh{} donnée une note de difficulté à l\PYGZsq{}exercice}
    \PYG{n}{correction} \PYG{o}{=} \PYG{n}{models}\PYG{o}{.}\PYG{n}{CharField}\PYG{p}{(}\PYG{n}{max\PYGZus{}length} \PYG{o}{=} \PYG{l+m+mi}{200}\PYG{p}{)} \PYG{c}{\PYGZsh{} corrigé de l\PYGZsq{}exercice ( obligatoire )}
    \PYG{k}{def} \PYG{n+nf}{\PYGZus{}\PYGZus{}str\PYGZus{}\PYGZus{}}\PYG{p}{(}\PYG{n+nb+bp}{self}\PYG{p}{)}\PYG{p}{:}
        \PYG{k}{return} \PYG{n+nb+bp}{self}\PYG{o}{.}\PYG{n}{title}
        
\PYG{c}{\PYGZsh{}}
\PYG{c}{\PYGZsh{} Modèle de Benoit pour les quiz}
\PYG{c}{\PYGZsh{}}

\PYG{k}{class} \PYG{n+nc}{Quiz}\PYG{p}{(}\PYG{n}{models}\PYG{o}{.}\PYG{n}{Model}\PYG{p}{)}\PYG{p}{:} \PYG{c}{\PYGZsh{}Infos générales sur le quiz}
    \PYG{n}{title} \PYG{o}{=} \PYG{n}{models}\PYG{o}{.}\PYG{n}{CharField}\PYG{p}{(}\PYG{n}{max\PYGZus{}length}\PYG{o}{=}\PYG{l+m+mi}{100}\PYG{p}{)}
    \PYG{n}{creation\PYGZus{}date} \PYG{o}{=} \PYG{n}{models}\PYG{o}{.}\PYG{n}{DateTimeField}\PYG{p}{(}\PYG{n}{auto\PYGZus{}now\PYGZus{}add}\PYG{o}{=}\PYG{n+nb+bp}{True}\PYG{p}{)}
    \PYG{n}{code} \PYG{o}{=} \PYG{n}{models}\PYG{o}{.}\PYG{n}{CharField}\PYG{p}{(}\PYG{n}{max\PYGZus{}length}\PYG{o}{=}\PYG{l+m+mi}{1000}\PYG{p}{)} \PYG{c}{\PYGZsh{}Format texte du quiz}
    \PYG{n}{author} \PYG{o}{=} \PYG{n}{models}\PYG{o}{.}\PYG{n}{ForeignKey}\PYG{p}{(}\PYG{n}{Teacher}\PYG{p}{)}
    \PYG{c}{\PYGZsh{}id\PYGZus{}chapter = models.ForeignKey(\PYGZsq{}teachers.Chapter\PYGZsq{})}
    
    \PYG{k}{def} \PYG{n+nf}{\PYGZus{}\PYGZus{}str\PYGZus{}\PYGZus{}}\PYG{p}{(}\PYG{n+nb+bp}{self}\PYG{p}{)}\PYG{p}{:}
        \PYG{k}{return} \PYG{n+nb+bp}{self}\PYG{o}{.}\PYG{n}{title}
        
        
        

\PYG{c}{\PYGZsh{}Modèle pour les groupes}
\PYG{k}{class} \PYG{n+nc}{Group}\PYG{p}{(}\PYG{n}{models}\PYG{o}{.}\PYG{n}{Model}\PYG{p}{)}\PYG{p}{:}
    \PYG{n}{name} \PYG{o}{=} \PYG{n}{models}\PYG{o}{.}\PYG{n}{CharField}\PYG{p}{(}\PYG{n}{max\PYGZus{}length}\PYG{o}{=}\PYG{l+m+mi}{30}\PYG{p}{)}
    \PYG{n}{teacher} \PYG{o}{=} \PYG{n}{models}\PYG{o}{.}\PYG{n}{ManyToManyField}\PYG{p}{(}\PYG{n}{Teacher}\PYG{p}{,} \PYG{n}{through}\PYG{o}{=}\PYG{l+s}{\PYGZsq{}}\PYG{l+s}{GroupMembers}\PYG{l+s}{\PYGZsq{}}\PYG{p}{)}
    \PYG{n}{student} \PYG{o}{=} \PYG{n}{models}\PYG{o}{.}\PYG{n}{ManyToManyField}\PYG{p}{(}\PYG{n}{Student}\PYG{p}{,} \PYG{n}{through} \PYG{o}{=} \PYG{l+s}{\PYGZsq{}}\PYG{l+s}{GroupMembers}\PYG{l+s}{\PYGZsq{}}\PYG{p}{)}
    \PYG{c}{\PYGZsh{}uniquement les devoirs exercices}
    \PYG{n}{homeworkExercise} \PYG{o}{=} \PYG{n}{models}\PYG{o}{.}\PYG{n}{ManyToManyField}\PYG{p}{(}\PYG{n}{Exercise}\PYG{p}{,} \PYG{n}{through} \PYG{o}{=} \PYG{l+s}{\PYGZsq{}}\PYG{l+s}{AssignHomework}\PYG{l+s}{\PYGZsq{}}\PYG{p}{)}
    \PYG{c}{\PYGZsh{}uniquement les devoirs quiz}
    \PYG{n}{homeworkCourse} \PYG{o}{=} \PYG{n}{models}\PYG{o}{.}\PYG{n}{ManyToManyField}\PYG{p}{(}\PYG{n}{Course}\PYG{p}{,} \PYG{n}{through} \PYG{o}{=} \PYG{l+s}{\PYGZsq{}}\PYG{l+s}{AssignHomework}\PYG{l+s}{\PYGZsq{}}\PYG{p}{)} 
    \PYG{c}{\PYGZsh{}uniquement les devoirs cours}
    \PYG{n}{homeworkQuiz} \PYG{o}{=} \PYG{n}{models}\PYG{o}{.}\PYG{n}{ManyToManyField}\PYG{p}{(}\PYG{n}{Quiz}\PYG{p}{,} \PYG{n}{through} \PYG{o}{=} \PYG{l+s}{\PYGZsq{}}\PYG{l+s}{AssignHomework}\PYG{l+s}{\PYGZsq{}}\PYG{p}{)} 
    \PYG{n}{created\PYGZus{}on} \PYG{o}{=} \PYG{n}{models}\PYG{o}{.}\PYG{n}{DateTimeField}\PYG{p}{(}\PYG{n}{auto\PYGZus{}now}\PYG{o}{=}\PYG{n+nb+bp}{True}\PYG{p}{)}
    
    \PYG{k}{def} \PYG{n+nf}{\PYGZus{}\PYGZus{}str\PYGZus{}\PYGZus{}}\PYG{p}{(}\PYG{n+nb+bp}{self}\PYG{p}{)}\PYG{p}{:}
        \PYG{k}{return}\PYG{l+s}{\PYGZdq{}}\PYG{l+s}{Classe \PYGZob{}0\PYGZcb{}}\PYG{l+s}{\PYGZdq{}}\PYG{o}{.}\PYG{n}{format}\PYG{p}{(}\PYG{n+nb+bp}{self}\PYG{o}{.}\PYG{n}{name}\PYG{p}{)}


\PYG{c}{\PYGZsh{}Table intermédiaire pour affecter un membre à un groupe}

\PYG{k}{class} \PYG{n+nc}{GroupMembers}\PYG{p}{(}\PYG{n}{models}\PYG{o}{.}\PYG{n}{Model}\PYG{p}{)}\PYG{p}{:}
    \PYG{n}{teacher} \PYG{o}{=} \PYG{n}{models}\PYG{o}{.}\PYG{n}{ForeignKey}\PYG{p}{(}\PYG{n}{Teacher}\PYG{p}{,} \PYG{n}{null} \PYG{o}{=} \PYG{n+nb+bp}{True}\PYG{p}{)}
    \PYG{n}{student} \PYG{o}{=} \PYG{n}{models}\PYG{o}{.}\PYG{n}{ForeignKey}\PYG{p}{(}\PYG{n}{Student}\PYG{p}{,} \PYG{n}{null} \PYG{o}{=} \PYG{n+nb+bp}{True}\PYG{p}{)}
    \PYG{n}{group} \PYG{o}{=} \PYG{n}{models}\PYG{o}{.}\PYG{n}{ForeignKey}\PYG{p}{(}\PYG{n}{Group}\PYG{p}{)}
    \PYG{n}{added\PYGZus{}on} \PYG{o}{=} \PYG{n}{models}\PYG{o}{.}\PYG{n}{DateTimeField}\PYG{p}{(}\PYG{n}{auto\PYGZus{}now}\PYG{o}{=}\PYG{n+nb+bp}{True}\PYG{p}{)}
    
\PYG{c}{\PYGZsh{}Table intermédiaire pour assigner un devoir à un groupe    }
    
\PYG{k}{class} \PYG{n+nc}{AssignHomework}\PYG{p}{(}\PYG{n}{models}\PYG{o}{.}\PYG{n}{Model}\PYG{p}{)}\PYG{p}{:}
    \PYG{n}{group} \PYG{o}{=} \PYG{n}{models}\PYG{o}{.}\PYG{n}{ForeignKey}\PYG{p}{(}\PYG{n}{Group}\PYG{p}{)}
    \PYG{n}{exercise} \PYG{o}{=} \PYG{n}{models}\PYG{o}{.}\PYG{n}{ForeignKey}\PYG{p}{(}\PYG{n}{Exercise}\PYG{p}{,} \PYG{n}{null} \PYG{o}{=} \PYG{n+nb+bp}{True}\PYG{p}{)}
    \PYG{n}{quiz} \PYG{o}{=} \PYG{n}{models}\PYG{o}{.}\PYG{n}{ForeignKey}\PYG{p}{(}\PYG{n}{Quiz}\PYG{p}{,} \PYG{n}{null} \PYG{o}{=} \PYG{n+nb+bp}{True}\PYG{p}{)}
    \PYG{n}{course} \PYG{o}{=} \PYG{n}{models}\PYG{o}{.}\PYG{n}{ForeignKey}\PYG{p}{(}\PYG{n}{Course}\PYG{p}{,} \PYG{n}{null} \PYG{o}{=} \PYG{n+nb+bp}{True}\PYG{p}{)}
    \PYG{n}{assigned\PYGZus{}on} \PYG{o}{=} \PYG{n}{models}\PYG{o}{.}\PYG{n}{DateTimeField}\PYG{p}{(}\PYG{n}{auto\PYGZus{}now}\PYG{o}{=}\PYG{n+nb+bp}{True}\PYG{p}{)}
    \PYG{n}{date} \PYG{o}{=} \PYG{n}{models}\PYG{o}{.}\PYG{n}{CharField}\PYG{p}{(}\PYG{n}{max\PYGZus{}length} \PYG{o}{=} \PYG{l+m+mi}{20}\PYG{p}{,} \PYG{n}{default} \PYG{o}{=} \PYG{l+s}{\PYGZdq{}}\PYG{l+s}{Demain}\PYG{l+s}{\PYGZdq{}}\PYG{p}{)}
\end{Verbatim}


\subsection{\texttt{views.py}}
\label{annexes:views-py}
\begin{Verbatim}[commandchars=\\\{\},numbers=left,firstnumber=1,stepnumber=1]
\PYG{k+kn}{from} \PYG{n+nn}{django.shortcuts} \PYG{k+kn}{import} \PYG{n}{render}\PYG{p}{,} \PYG{n}{redirect}
\PYG{k+kn}{from} \PYG{n+nn}{django.contrib.auth} \PYG{k+kn}{import} \PYG{n}{authenticate}\PYG{p}{,} \PYG{n}{login}\PYG{p}{,} \PYG{n}{logout}
\PYG{k+kn}{from} \PYG{n+nn}{dashboard.forms} \PYG{k+kn}{import} \PYG{n}{NewGroupForm}\PYG{p}{,} \PYG{n}{NewStudentForm}\PYG{p}{,} \PYG{n}{NewTeacherForm}\PYG{p}{,}\PYGZbs{}
                            \PYG{n}{AddHomeworkForm}\PYG{p}{,} \PYG{n}{NewPasswordForm}
\PYG{k+kn}{from} \PYG{n+nn}{django.core.urlresolvers} \PYG{k+kn}{import} \PYG{n}{reverse}
\PYG{k+kn}{from} \PYG{n+nn}{common.models} \PYG{k+kn}{import} \PYG{n}{Group}\PYG{p}{,} \PYG{n}{Teacher}\PYG{p}{,} \PYG{n}{GroupMembers}\PYG{p}{,} \PYG{n}{Student}\PYG{p}{,} \PYGZbs{}
                            \PYG{n}{AssignHomework}\PYG{p}{,} \PYG{n}{Exercise}\PYG{p}{,} \PYG{n}{Quiz}\PYG{p}{,} \PYG{n}{Course}
\PYG{k+kn}{from} \PYG{n+nn}{django.contrib.auth.models} \PYG{k+kn}{import} \PYG{n}{User}
\PYG{k+kn}{from} \PYG{n+nn}{django.http} \PYG{k+kn}{import} \PYG{n}{HttpResponse}

\PYG{c}{\PYGZsh{}Accueil du dashboard}
\PYG{k}{def} \PYG{n+nf}{home}\PYG{p}{(}\PYG{n}{request}\PYG{p}{)}\PYG{p}{:}
    \PYG{n}{voyelle} \PYG{o}{=} \PYG{l+s}{\PYGZsq{}}\PYG{l+s}{aeiouyàäâéèëêîïíìôöõòûüùúAEIOUY}\PYG{l+s}{\PYGZsq{}} \PYG{c}{\PYGZsh{}Pour déterminer si le template affiche De ou D\PYGZsq{}}
    \PYG{n}{user} \PYG{o}{=} \PYG{n}{Teacher}\PYG{o}{.}\PYG{n}{objects}\PYG{o}{.}\PYG{n}{get}\PYG{p}{(}\PYG{n}{user} \PYG{o}{=} \PYG{n}{request}\PYG{o}{.}\PYG{n}{user}\PYG{p}{)}
    \PYG{n}{firstLetter} \PYG{o}{=} \PYG{n}{request}\PYG{o}{.}\PYG{n}{user}\PYG{o}{.}\PYG{n}{username}\PYG{p}{[}\PYG{l+m+mi}{0}\PYG{p}{]}\PYG{c}{\PYGZsh{}Idem}
    \PYG{k}{return} \PYG{n}{render}\PYG{p}{(}\PYG{n}{request}\PYG{p}{,} \PYG{l+s}{\PYGZsq{}}\PYG{l+s}{dashboard/templates/dashboard/index.html}\PYG{l+s}{\PYGZsq{}}\PYG{p}{,} \PYG{n+nb}{locals}\PYG{p}{(}\PYG{p}{)}\PYG{p}{)}

\PYG{c}{\PYGZsh{}Exercices, quiz et cours}
\PYG{k}{def} \PYG{n+nf}{exercises}\PYG{p}{(}\PYG{n}{request}\PYG{p}{)}\PYG{p}{:}
    \PYG{n}{user} \PYG{o}{=} \PYG{n}{Teacher}\PYG{o}{.}\PYG{n}{objects}\PYG{o}{.}\PYG{n}{get}\PYG{p}{(}\PYG{n}{user} \PYG{o}{=} \PYG{n}{request}\PYG{o}{.}\PYG{n}{user}\PYG{p}{)}
    \PYG{k}{return} \PYG{n}{render}\PYG{p}{(}\PYG{n}{request}\PYG{p}{,} \PYG{l+s}{\PYGZsq{}}\PYG{l+s}{dashboard/templates/dashboard/exercises.html}\PYG{l+s}{\PYGZsq{}}\PYG{p}{,} \PYG{n+nb}{locals}\PYG{p}{(}\PYG{p}{)}\PYG{p}{)}

\PYG{c}{\PYGZsh{}Création de groupe}
\PYG{k}{def} \PYG{n+nf}{newgroup}\PYG{p}{(}\PYG{n}{request}\PYG{p}{)}\PYG{p}{:}
    \PYG{n}{success} \PYG{o}{=} \PYG{n+nb+bp}{False}
    \PYG{n}{user} \PYG{o}{=} \PYG{n}{Teacher}\PYG{o}{.}\PYG{n}{objects}\PYG{o}{.}\PYG{n}{get}\PYG{p}{(}\PYG{n}{user} \PYG{o}{=} \PYG{n}{request}\PYG{o}{.}\PYG{n}{user}\PYG{p}{)}
    \PYG{k}{if} \PYG{n}{request}\PYG{o}{.}\PYG{n}{method} \PYG{o}{==} \PYG{l+s}{\PYGZdq{}}\PYG{l+s}{POST}\PYG{l+s}{\PYGZdq{}}\PYG{p}{:}
        \PYG{n}{form} \PYG{o}{=} \PYG{n}{NewGroupForm}\PYG{p}{(}\PYG{n}{request}\PYG{o}{.}\PYG{n}{POST}\PYG{p}{)}
        \PYG{k}{if} \PYG{n}{form}\PYG{o}{.}\PYG{n}{is\PYGZus{}valid}\PYG{p}{(}\PYG{p}{)}\PYG{p}{:}
            \PYG{n}{group\PYGZus{}name} \PYG{o}{=} \PYG{n}{form}\PYG{o}{.}\PYG{n}{cleaned\PYGZus{}data}\PYG{p}{[}\PYG{l+s}{\PYGZdq{}}\PYG{l+s}{group\PYGZus{}name}\PYG{l+s}{\PYGZdq{}}\PYG{p}{]}
            
            \PYG{n}{newGroup} \PYG{o}{=} \PYG{n}{Group}\PYG{o}{.}\PYG{n}{objects}\PYG{o}{.}\PYG{n}{create}\PYG{p}{(}\PYG{n}{name} \PYG{o}{=} \PYG{n}{group\PYGZus{}name}\PYG{p}{)}
            \PYG{n}{newGroup}\PYG{o}{.}\PYG{n}{save}\PYG{p}{(}\PYG{p}{)}
            \PYG{c}{\PYGZsh{}Lie le Teacher et le groupe à travers la table intermédiaire}
            \PYG{n}{teacherToGroup} \PYG{o}{=} \PYG{n}{GroupMembers}\PYG{p}{(}\PYG{n}{teacher} \PYG{o}{=} \PYG{n}{user}\PYG{p}{,} \PYG{n}{group} \PYG{o}{=} \PYG{n}{newGroup}\PYG{p}{)} 
            \PYG{n}{teacherToGroup}\PYG{o}{.}\PYG{n}{save}\PYG{p}{(}\PYG{p}{)}
            \PYG{n}{success} \PYG{o}{=} \PYG{n+nb+bp}{True} \PYG{c}{\PYGZsh{}Pour retourner le message de confirmation}
    \PYG{k}{else}\PYG{p}{:}
        \PYG{n}{form} \PYG{o}{=} \PYG{n}{NewGroupForm}\PYG{p}{(}\PYG{p}{)}
    \PYG{k}{return} \PYG{n}{render}\PYG{p}{(}\PYG{n}{request}\PYG{p}{,} \PYG{l+s}{\PYGZdq{}}\PYG{l+s}{dashboard/templates/dashboard/newclass.html}\PYG{l+s}{\PYGZdq{}}\PYG{p}{,} \PYG{n+nb}{locals}\PYG{p}{(}\PYG{p}{)}\PYG{p}{)}

\PYG{c}{\PYGZsh{}Changement de mot de passe}
\PYG{k}{def} \PYG{n+nf}{profil}\PYG{p}{(}\PYG{n}{request}\PYG{p}{)}\PYG{p}{:}
    \PYG{n}{user} \PYG{o}{=} \PYG{n}{Teacher}\PYG{o}{.}\PYG{n}{objects}\PYG{o}{.}\PYG{n}{get}\PYG{p}{(}\PYG{n}{user} \PYG{o}{=} \PYG{n}{request}\PYG{o}{.}\PYG{n}{user}\PYG{p}{)}
    \PYG{n}{success} \PYG{o}{=} \PYG{l+s}{\PYGZsq{}}\PYG{l+s}{\PYGZsq{}}
    \PYG{k}{if} \PYG{n}{request}\PYG{o}{.}\PYG{n}{method} \PYG{o}{==} \PYG{l+s}{\PYGZdq{}}\PYG{l+s}{POST}\PYG{l+s}{\PYGZdq{}}\PYG{p}{:}
        \PYG{n}{userProfile} \PYG{o}{=} \PYG{n}{request}\PYG{o}{.}\PYG{n}{user}
        \PYG{n}{form} \PYG{o}{=} \PYG{n}{NewPasswordForm}\PYG{p}{(}\PYG{n}{request}\PYG{o}{.}\PYG{n}{POST}\PYG{p}{)}
        \PYG{k}{if} \PYG{n}{form}\PYG{o}{.}\PYG{n}{is\PYGZus{}valid}\PYG{p}{(}\PYG{p}{)}\PYG{p}{:}
            \PYG{n}{password} \PYG{o}{=} \PYG{n}{form}\PYG{o}{.}\PYG{n}{cleaned\PYGZus{}data}\PYG{p}{[}\PYG{l+s}{\PYGZdq{}}\PYG{l+s}{password}\PYG{l+s}{\PYGZdq{}}\PYG{p}{]}
            \PYG{n}{passwordConfirm} \PYG{o}{=} \PYG{n}{form}\PYG{o}{.}\PYG{n}{cleaned\PYGZus{}data}\PYG{p}{[}\PYG{l+s}{\PYGZdq{}}\PYG{l+s}{passwordConfirm}\PYG{l+s}{\PYGZdq{}}\PYG{p}{]}
            \PYG{k}{if} \PYG{n}{password} \PYG{o}{!=} \PYG{n}{passwordConfirm}\PYG{p}{:}
                \PYG{n}{success} \PYG{o}{=} \PYG{n+nb+bp}{False} \PYG{c}{\PYGZsh{}Message d\PYGZsq{}erreur}
            \PYG{k}{else}\PYG{p}{:}
                \PYG{n}{success} \PYG{o}{=} \PYG{n+nb+bp}{True} \PYG{c}{\PYGZsh{}Message de confirmation}
                \PYG{n}{u} \PYG{o}{=} \PYG{n}{request}\PYG{o}{.}\PYG{n}{user}
                \PYG{n}{u}\PYG{o}{.}\PYG{n}{set\PYGZus{}password}\PYG{p}{(}\PYG{n}{password}\PYG{p}{)}
                \PYG{n}{u}\PYG{o}{.}\PYG{n}{save}\PYG{p}{(}\PYG{p}{)}
    \PYG{k}{else}\PYG{p}{:}
        \PYG{n}{form} \PYG{o}{=} \PYG{n}{NewPasswordForm}\PYG{p}{(}\PYG{p}{)}
    \PYG{k}{return} \PYG{n}{render}\PYG{p}{(}\PYG{n}{request}\PYG{p}{,} \PYG{l+s}{\PYGZsq{}}\PYG{l+s}{dashboard/templates/dashboard/profile.html}\PYG{l+s}{\PYGZsq{}}\PYG{p}{,} \PYG{n+nb}{locals}\PYG{p}{(}\PYG{p}{)}\PYG{p}{)}
    
\PYG{k}{def} \PYG{n+nf}{group}\PYG{p}{(}\PYG{n}{request}\PYG{p}{,} \PYG{n}{group\PYGZus{}id}\PYG{p}{)}\PYG{p}{:}
    \PYG{n}{user} \PYG{o}{=} \PYG{n}{Teacher}\PYG{o}{.}\PYG{n}{objects}\PYG{o}{.}\PYG{n}{get}\PYG{p}{(}\PYG{n}{user} \PYG{o}{=} \PYG{n}{request}\PYG{o}{.}\PYG{n}{user}\PYG{p}{)}
    \PYG{n}{group} \PYG{o}{=} \PYG{n}{Group}\PYG{o}{.}\PYG{n}{objects}\PYG{o}{.}\PYG{n}{get}\PYG{p}{(}\PYG{n+nb}{id} \PYG{o}{=} \PYG{n}{group\PYGZus{}id}\PYG{p}{)}
    
    \PYG{n}{studentList} \PYG{o}{=} \PYG{n}{group}\PYG{o}{.}\PYG{n}{student}\PYG{o}{.}\PYG{n}{all}\PYG{p}{(}\PYG{p}{)}
    \PYG{n}{teacherList} \PYG{o}{=} \PYG{n}{group}\PYG{o}{.}\PYG{n}{teacher}\PYG{o}{.}\PYG{n}{all}\PYG{p}{(}\PYG{p}{)}
    \PYG{n}{homeworkExList} \PYG{o}{=} \PYG{n}{group}\PYG{o}{.}\PYG{n}{homeworkExercise}\PYG{o}{.}\PYG{n}{all}\PYG{p}{(}\PYG{p}{)}\PYG{c}{\PYGZsh{}}
    \PYG{n}{homeworkQuList} \PYG{o}{=} \PYG{n}{group}\PYG{o}{.}\PYG{n}{homeworkQuiz}\PYG{o}{.}\PYG{n}{all}\PYG{p}{(}\PYG{p}{)}    \PYG{c}{\PYGZsh{} Pour avoir la liste des devoirs}
    \PYG{n}{homeworkCoList} \PYG{o}{=} \PYG{n}{group}\PYG{o}{.}\PYG{n}{homeworkCourse}\PYG{o}{.}\PYG{n}{all}\PYG{p}{(}\PYG{p}{)}  \PYG{c}{\PYGZsh{} selon les genres d\PYGZsq{}activités}

    \PYG{n}{deleteConfirmation} \PYG{o}{=} \PYG{n+nb+bp}{False} \PYG{c}{\PYGZsh{}Pour supprimer une classe}
    
    \PYG{k}{if} \PYG{n}{request}\PYG{o}{.}\PYG{n}{method} \PYG{o}{==} \PYG{l+s}{\PYGZdq{}}\PYG{l+s}{POST}\PYG{l+s}{\PYGZdq{}}\PYG{p}{:}
        
        \PYG{c}{\PYGZsh{}Ajouter un professeur au groupe}
        \PYG{k}{if} \PYG{l+s}{\PYGZsq{}}\PYG{l+s}{addTeacher}\PYG{l+s}{\PYGZsq{}} \PYG{o+ow}{in} \PYG{n}{request}\PYG{o}{.}\PYG{n}{POST}\PYG{p}{:}
            \PYG{n}{erreurTeacher} \PYG{o}{=} \PYG{n+nb+bp}{False}
            \PYG{n}{formTeacher} \PYG{o}{=} \PYG{n}{NewTeacherForm}\PYG{p}{(}\PYG{n}{request}\PYG{o}{.}\PYG{n}{POST}\PYG{p}{)}
            \PYG{k}{if} \PYG{n}{formTeacher}\PYG{o}{.}\PYG{n}{is\PYGZus{}valid}\PYG{p}{(}\PYG{p}{)}\PYG{p}{:}
                \PYG{n}{newTeacher} \PYG{o}{=} \PYG{n}{formTeacher}\PYG{o}{.}\PYG{n}{cleaned\PYGZus{}data}\PYG{p}{[}\PYG{l+s}{\PYGZdq{}}\PYG{l+s}{nickname}\PYG{l+s}{\PYGZdq{}}\PYG{p}{]}
                \PYG{k}{try}\PYG{p}{:}
                    \PYG{k}{try}\PYG{p}{:}
                        \PYG{n}{teacherUser} \PYG{o}{=} \PYG{n}{User}\PYG{o}{.}\PYG{n}{objects}\PYG{o}{.}\PYG{n}{get}\PYG{p}{(}\PYG{n}{username} \PYG{o}{=} \PYG{n}{newTeacher}\PYG{p}{)}
                        \PYG{n}{teacher} \PYG{o}{=} \PYG{n}{Teacher}\PYG{o}{.}\PYG{n}{objects}\PYG{o}{.}\PYG{n}{get}\PYG{p}{(}\PYG{n}{user} \PYG{o}{=} \PYG{n}{teacherUser}\PYG{p}{)}
                        \PYG{n}{newTeacherToGroup} \PYG{o}{=} \PYG{n}{GroupMembers}\PYG{p}{(}\PYG{n}{teacher} \PYG{o}{=} \PYG{n}{teacher}\PYG{p}{,} \PYG{n}{group} \PYG{o}{=} \PYG{n}{group}\PYG{p}{)}
                        \PYG{n}{newTeacherToGroup}\PYG{o}{.}\PYG{n}{save}\PYG{p}{(}\PYG{p}{)}
                    \PYG{k}{except} \PYG{n}{User}\PYG{o}{.}\PYG{n}{DoesNotExist}\PYG{p}{:}
                        \PYG{n}{erreurTeacher} \PYG{o}{=} \PYG{n+nb+bp}{True} \PYG{c}{\PYGZsh{}Message d\PYGZsq{}erreur}
                \PYG{k}{except} \PYG{n}{Teacher}\PYG{o}{.}\PYG{n}{DoesNotExist}\PYG{p}{:}
                    \PYG{n}{erreurTeacher} \PYG{o}{=} \PYG{n+nb+bp}{True} \PYG{c}{\PYGZsh{}Idem}
                    
        \PYG{c}{\PYGZsh{}Ajouter un élève au groupe       }
        \PYG{k}{elif} \PYG{l+s}{\PYGZsq{}}\PYG{l+s}{addStudent}\PYG{l+s}{\PYGZsq{}} \PYG{o+ow}{in} \PYG{n}{request}\PYG{o}{.}\PYG{n}{POST}\PYG{p}{:}
            \PYG{n}{formStudent} \PYG{o}{=} \PYG{n}{NewStudentForm}\PYG{p}{(}\PYG{n}{request}\PYG{o}{.}\PYG{n}{POST}\PYG{p}{)}
            \PYG{n}{erreurStudent} \PYG{o}{=} \PYG{n+nb+bp}{False}
            \PYG{k}{if} \PYG{n}{formStudent}\PYG{o}{.}\PYG{n}{is\PYGZus{}valid}\PYG{p}{(}\PYG{p}{)}\PYG{p}{:}
                \PYG{k}{try}\PYG{p}{:}
                    \PYG{k}{try}\PYG{p}{:}
                        \PYG{n}{newStudent} \PYG{o}{=} \PYG{n}{formStudent}\PYG{o}{.}\PYG{n}{cleaned\PYGZus{}data}\PYG{p}{[}\PYG{l+s}{\PYGZdq{}}\PYG{l+s}{nickname}\PYG{l+s}{\PYGZdq{}}\PYG{p}{]}
                        \PYG{n}{studentUser} \PYG{o}{=} \PYG{n}{User}\PYG{o}{.}\PYG{n}{objects}\PYG{o}{.}\PYG{n}{get}\PYG{p}{(}\PYG{n}{username} \PYG{o}{=} \PYG{n}{newStudent}\PYG{p}{)}
                        \PYG{n}{student} \PYG{o}{=} \PYG{n}{Student}\PYG{o}{.}\PYG{n}{objects}\PYG{o}{.}\PYG{n}{get}\PYG{p}{(}\PYG{n}{user} \PYG{o}{=} \PYG{n}{studentUser}\PYG{p}{)}
                        \PYG{n}{newStudentToGroup} \PYG{o}{=} \PYG{n}{GroupMembers}\PYG{p}{(}\PYG{n}{student} \PYG{o}{=} \PYG{n}{student}\PYG{p}{,} \PYG{n}{group} \PYG{o}{=} \PYG{n}{group}\PYG{p}{)}
                        \PYG{n}{newStudentToGroup}\PYG{o}{.}\PYG{n}{save}\PYG{p}{(}\PYG{p}{)}
                    \PYG{k}{except} \PYG{n}{User}\PYG{o}{.}\PYG{n}{DoesNotExist}\PYG{p}{:}
                        \PYG{n}{erreurStudent} \PYG{o}{=} \PYG{n+nb+bp}{True} \PYG{c}{\PYGZsh{}Message d\PYGZsq{}erreur}
                \PYG{k}{except} \PYG{n}{Student}\PYG{o}{.}\PYG{n}{DoesNotExist}\PYG{p}{:}
                    \PYG{n}{erreurStudent} \PYG{o}{=} \PYG{n+nb+bp}{True} \PYG{c}{\PYGZsh{}Idem}
                    
                    
        \PYG{c}{\PYGZsh{}Assigner un devoir}
        \PYG{k}{elif} \PYG{l+s}{\PYGZsq{}}\PYG{l+s}{assignHomework}\PYG{l+s}{\PYGZsq{}} \PYG{o+ow}{in} \PYG{n}{request}\PYG{o}{.}\PYG{n}{POST}\PYG{p}{:}
            \PYG{n}{formHomework} \PYG{o}{=} \PYG{n}{AddHomeworkForm}\PYG{p}{(}\PYG{n}{request}\PYG{o}{.}\PYG{n}{POST}\PYG{p}{)}
            \PYG{n}{erreur} \PYG{o}{=} \PYG{n+nb+bp}{False}
            \PYG{k}{if} \PYG{n}{formHomework}\PYG{o}{.}\PYG{n}{is\PYGZus{}valid}\PYG{p}{(}\PYG{p}{)}\PYG{p}{:}
                \PYG{n}{homeworkid} \PYG{o}{=} \PYG{n}{formHomework}\PYG{o}{.}\PYG{n}{cleaned\PYGZus{}data}\PYG{p}{[}\PYG{l+s}{\PYGZdq{}}\PYG{l+s}{homeworkid}\PYG{l+s}{\PYGZdq{}}\PYG{p}{]}
                \PYG{n}{genre} \PYG{o}{=} \PYG{n}{formHomework}\PYG{o}{.}\PYG{n}{cleaned\PYGZus{}data}\PYG{p}{[}\PYG{l+s}{\PYGZdq{}}\PYG{l+s}{genre}\PYG{l+s}{\PYGZdq{}}\PYG{p}{]}
                
                \PYG{c}{\PYGZsh{}Cherche l\PYGZsq{}activité selon le genre choisi}
                \PYG{k}{if} \PYG{n}{genre} \PYG{o}{==} \PYG{l+s}{\PYGZdq{}}\PYG{l+s}{exercise}\PYG{l+s}{\PYGZdq{}}\PYG{p}{:}
                    \PYG{k}{try}\PYG{p}{:}
                        \PYG{n}{exercise} \PYG{o}{=} \PYG{n}{Exercise}\PYG{o}{.}\PYG{n}{objects}\PYG{o}{.}\PYG{n}{get}\PYG{p}{(}\PYG{n+nb}{id} \PYG{o}{=} \PYG{n}{homeworkid}\PYG{p}{)}
                        \PYG{n}{newHomework} \PYG{o}{=} \PYG{n}{AssignHomework}\PYG{p}{(}\PYG{n}{exercise} \PYG{o}{=} \PYG{n}{exercise}\PYG{p}{,} \PYG{n}{group} \PYG{o}{=} \PYG{n}{group}\PYG{p}{)}
                        \PYG{n}{newHomework}\PYG{o}{.}\PYG{n}{save}\PYG{p}{(}\PYG{p}{)}
                    \PYG{k}{except} \PYG{n}{Exercise}\PYG{o}{.}\PYG{n}{DoesNotExist}\PYG{p}{:}
                        \PYG{n}{erreur} \PYG{o}{=} \PYG{n+nb+bp}{True} \PYG{c}{\PYGZsh{}Message d\PYGZsq{}erreur}
                
                \PYG{k}{if} \PYG{n}{genre} \PYG{o}{==} \PYG{l+s}{\PYGZdq{}}\PYG{l+s}{quiz}\PYG{l+s}{\PYGZdq{}}\PYG{p}{:}
                    \PYG{k}{try}\PYG{p}{:}
                        \PYG{n}{quiz} \PYG{o}{=} \PYG{n}{Quiz}\PYG{o}{.}\PYG{n}{objects}\PYG{o}{.}\PYG{n}{get}\PYG{p}{(}\PYG{n+nb}{id} \PYG{o}{=} \PYG{n}{homeworkid}\PYG{p}{)}
                        \PYG{n}{newHomework} \PYG{o}{=} \PYG{n}{AssignHomework}\PYG{p}{(}\PYG{n}{quiz} \PYG{o}{=} \PYG{n}{quiz}\PYG{p}{,} \PYG{n}{group} \PYG{o}{=} \PYG{n}{group}\PYG{p}{)}
                        \PYG{n}{newHomework}\PYG{o}{.}\PYG{n}{save}\PYG{p}{(}\PYG{p}{)}
                    \PYG{k}{except} \PYG{n}{Quiz}\PYG{o}{.}\PYG{n}{DoesNotExist}\PYG{p}{:}
                        \PYG{n}{erreur} \PYG{o}{=} \PYG{n+nb+bp}{True} \PYG{c}{\PYGZsh{}Idem}
                
                \PYG{k}{if} \PYG{n}{genre} \PYG{o}{==} \PYG{l+s}{\PYGZdq{}}\PYG{l+s}{course}\PYG{l+s}{\PYGZdq{}}\PYG{p}{:}
                    \PYG{k}{try}\PYG{p}{:}
                        \PYG{n}{cours} \PYG{o}{=} \PYG{n}{Course}\PYG{o}{.}\PYG{n}{objects}\PYG{o}{.}\PYG{n}{get}\PYG{p}{(}\PYG{n+nb}{id} \PYG{o}{=} \PYG{n}{homeworkid}\PYG{p}{)}
                        \PYG{n}{newHomework} \PYG{o}{=} \PYG{n}{AssignHomework}\PYG{p}{(}\PYG{n}{course} \PYG{o}{=} \PYG{n}{cours}\PYG{p}{,} \PYG{n}{group} \PYG{o}{=} \PYG{n}{group}\PYG{p}{)}
                        \PYG{n}{newHomework}\PYG{o}{.}\PYG{n}{save}\PYG{p}{(}\PYG{p}{)}
                    \PYG{k}{except} \PYG{n}{Course}\PYG{o}{.}\PYG{n}{DoesNotExist}\PYG{p}{:}
                        \PYG{n}{erreur} \PYG{o}{=} \PYG{n+nb+bp}{True} \PYG{c}{\PYGZsh{}Idem}
        \PYG{k}{elif} \PYG{l+s}{\PYGZsq{}}\PYG{l+s}{deleteClass}\PYG{l+s}{\PYGZsq{}} \PYG{o+ow}{in} \PYG{n}{request}\PYG{o}{.}\PYG{n}{POST}\PYG{p}{:}
            \PYG{n}{deleteConfirmation} \PYG{o}{=} \PYG{n+nb+bp}{True} \PYG{c}{\PYGZsh{}Fait apparaître le deuxième bouton de confirmation}
        
        \PYG{c}{\PYGZsh{}Supprime la classe}
        \PYG{k}{elif} \PYG{l+s}{\PYGZsq{}}\PYG{l+s}{deleteClassConfirm}\PYG{l+s}{\PYGZsq{}} \PYG{o+ow}{in} \PYG{n}{request}\PYG{o}{.}\PYG{n}{POST}\PYG{p}{:}
            \PYG{n}{group} \PYG{o}{=} \PYG{n}{Group}\PYG{o}{.}\PYG{n}{objects}\PYG{o}{.}\PYG{n}{get}\PYG{p}{(}\PYG{n+nb}{id} \PYG{o}{=} \PYG{n}{group\PYGZus{}id}\PYG{p}{)}
            \PYG{n}{group}\PYG{o}{.}\PYG{n}{delete}\PYG{p}{(}\PYG{p}{)}
            \PYG{k}{return} \PYG{n}{redirect}\PYG{p}{(}\PYG{l+s}{\PYGZsq{}}\PYG{l+s}{home}\PYG{l+s}{\PYGZsq{}}\PYG{p}{)}
            
        
        \PYG{n}{formStudent} \PYG{o}{=} \PYG{n}{NewStudentForm}\PYG{p}{(}\PYG{p}{)}
        \PYG{n}{formTeacher} \PYG{o}{=} \PYG{n}{NewTeacherForm}\PYG{p}{(}\PYG{p}{)}
        \PYG{n}{formHomework} \PYG{o}{=} \PYG{n}{AddHomeworkForm}\PYG{p}{(}\PYG{p}{)}
                    
    \PYG{k}{else}\PYG{p}{:}
        \PYG{n}{formStudent} \PYG{o}{=} \PYG{n}{NewStudentForm}\PYG{p}{(}\PYG{p}{)}
        \PYG{n}{formTeacher} \PYG{o}{=} \PYG{n}{NewTeacherForm}\PYG{p}{(}\PYG{p}{)}
        \PYG{n}{formHomework} \PYG{o}{=} \PYG{n}{AddHomeworkForm}\PYG{p}{(}\PYG{p}{)}
    \PYG{k}{return} \PYG{n}{render}\PYG{p}{(}\PYG{n}{request}\PYG{p}{,} \PYG{l+s}{\PYGZsq{}}\PYG{l+s}{dashboard/templates/dashboard/classe.html}\PYG{l+s}{\PYGZsq{}}\PYG{p}{,} \PYG{n+nb}{locals}\PYG{p}{(}\PYG{p}{)}\PYG{p}{)}

\PYG{c}{\PYGZsh{}Retirer d\PYGZsq{}un groupe}
\PYG{k}{def} \PYG{n+nf}{deleteFromGroup}\PYG{p}{(}\PYG{n}{request}\PYG{p}{,} \PYG{n}{member\PYGZus{}id}\PYG{p}{,} \PYG{n}{group\PYGZus{}id}\PYG{p}{)}\PYG{p}{:}
    \PYG{k}{if} \PYG{n}{request}\PYG{o}{.}\PYG{n}{method} \PYG{o}{==} \PYG{l+s}{\PYGZdq{}}\PYG{l+s}{POST}\PYG{l+s}{\PYGZdq{}}\PYG{p}{:}
        
        \PYG{c}{\PYGZsh{}Selon élève ou professeur}
        \PYG{k}{if} \PYG{l+s}{\PYGZsq{}}\PYG{l+s}{deleteStudent}\PYG{l+s}{\PYGZsq{}} \PYG{o+ow}{in} \PYG{n}{request}\PYG{o}{.}\PYG{n}{POST}\PYG{p}{:}
        
            \PYG{n}{student} \PYG{o}{=} \PYG{n}{Student}\PYG{o}{.}\PYG{n}{objects}\PYG{o}{.}\PYG{n}{get}\PYG{p}{(}\PYG{n+nb}{id} \PYG{o}{=} \PYG{n}{member\PYGZus{}id}\PYG{p}{)}
            \PYG{n}{group} \PYG{o}{=} \PYG{n}{Group}\PYG{o}{.}\PYG{n}{objects}\PYG{o}{.}\PYG{n}{get}\PYG{p}{(}\PYG{n+nb}{id} \PYG{o}{=} \PYG{n}{group\PYGZus{}id}\PYG{p}{)}
            \PYG{n}{studentToGroup} \PYG{o}{=} \PYG{n}{GroupMembers}\PYG{o}{.}\PYG{n}{objects}\PYG{o}{.}\PYG{n}{get}\PYG{p}{(}\PYG{n}{student} \PYG{o}{=} \PYG{n}{student}\PYG{p}{,} \PYG{n}{group} \PYG{o}{=} \PYG{n}{group}\PYG{p}{)}
            \PYG{n}{studentToGroup}\PYG{o}{.}\PYG{n}{delete}\PYG{p}{(}\PYG{p}{)}
            

        \PYG{k}{elif} \PYG{l+s}{\PYGZsq{}}\PYG{l+s}{deleteTeacher}\PYG{l+s}{\PYGZsq{}} \PYG{o+ow}{in} \PYG{n}{request}\PYG{o}{.}\PYG{n}{POST}\PYG{p}{:}
            \PYG{n}{teacher} \PYG{o}{=} \PYG{n}{Teacher}\PYG{o}{.}\PYG{n}{objects}\PYG{o}{.}\PYG{n}{get}\PYG{p}{(}\PYG{n+nb}{id} \PYG{o}{=} \PYG{n}{member\PYGZus{}id}\PYG{p}{)}
            \PYG{n}{group} \PYG{o}{=} \PYG{n}{Group}\PYG{o}{.}\PYG{n}{objects}\PYG{o}{.}\PYG{n}{get}\PYG{p}{(}\PYG{n+nb}{id} \PYG{o}{=} \PYG{n}{group\PYGZus{}id}\PYG{p}{)}
            \PYG{n}{teacherToGroup} \PYG{o}{=} \PYG{n}{GroupMembers}\PYG{o}{.}\PYG{n}{objects}\PYG{o}{.}\PYG{n}{get}\PYG{p}{(}\PYG{n}{teacher} \PYG{o}{=} \PYG{n}{teacher}\PYG{p}{,} \PYG{n}{group} \PYG{o}{=} \PYG{n}{group}\PYG{p}{)}
            \PYG{n}{teacherToGroup}\PYG{o}{.}\PYG{n}{delete}\PYG{p}{(}\PYG{p}{)}
            
    \PYG{k}{return} \PYG{n}{redirect}\PYG{p}{(}\PYG{l+s}{\PYGZsq{}}\PYG{l+s}{group\PYGZus{}view}\PYG{l+s}{\PYGZsq{}}\PYG{p}{,} \PYG{n}{group\PYGZus{}id} \PYG{o}{=} \PYG{n}{group\PYGZus{}id}\PYG{p}{)}

\PYG{c}{\PYGZsh{}Supprimer une activité}
\PYG{k}{def} \PYG{n+nf}{deleteActivity}\PYG{p}{(}\PYG{n}{request}\PYG{p}{,} \PYG{n}{activity\PYGZus{}id}\PYG{p}{)}\PYG{p}{:}
    \PYG{k}{if} \PYG{n}{request}\PYG{o}{.}\PYG{n}{method} \PYG{o}{==} \PYG{l+s}{\PYGZdq{}}\PYG{l+s}{POST}\PYG{l+s}{\PYGZdq{}}\PYG{p}{:}
        
        \PYG{c}{\PYGZsh{}Selon exercice, quiz ou cours}
        \PYG{k}{if} \PYG{l+s}{\PYGZsq{}}\PYG{l+s}{deleteExercise}\PYG{l+s}{\PYGZsq{}} \PYG{o+ow}{in} \PYG{n}{request}\PYG{o}{.}\PYG{n}{POST}\PYG{p}{:}
            \PYG{n}{exercise} \PYG{o}{=} \PYG{n}{Exercise}\PYG{o}{.}\PYG{n}{objects}\PYG{o}{.}\PYG{n}{get}\PYG{p}{(}\PYG{n+nb}{id} \PYG{o}{=} \PYG{n}{activity\PYGZus{}id}\PYG{p}{)}
            \PYG{n}{exercise}\PYG{o}{.}\PYG{n}{delete}\PYG{p}{(}\PYG{p}{)}
        \PYG{k}{if} \PYG{l+s}{\PYGZsq{}}\PYG{l+s}{deleteQuiz}\PYG{l+s}{\PYGZsq{}} \PYG{o+ow}{in} \PYG{n}{request}\PYG{o}{.}\PYG{n}{POST}\PYG{p}{:}
            \PYG{n}{quiz} \PYG{o}{=} \PYG{n}{Quiz}\PYG{o}{.}\PYG{n}{objects}\PYG{o}{.}\PYG{n}{get}\PYG{p}{(}\PYG{n+nb}{id} \PYG{o}{=} \PYG{n}{activity\PYGZus{}id}\PYG{p}{)}
            \PYG{n}{quiz}\PYG{o}{.}\PYG{n}{delete}\PYG{p}{(}\PYG{p}{)}
        \PYG{k}{if} \PYG{l+s}{\PYGZsq{}}\PYG{l+s}{deleteCourse}\PYG{l+s}{\PYGZsq{}} \PYG{o+ow}{in} \PYG{n}{request}\PYG{o}{.}\PYG{n}{POST}\PYG{p}{:}
            \PYG{n}{course} \PYG{o}{=} \PYG{n}{Course}\PYG{o}{.}\PYG{n}{objects}\PYG{o}{.}\PYG{n}{get}\PYG{p}{(}\PYG{n+nb}{id} \PYG{o}{=} \PYG{n}{activity\PYGZus{}id}\PYG{p}{)}
            \PYG{n}{course}\PYG{o}{.}\PYG{n}{delete}\PYG{p}{(}\PYG{p}{)}
    \PYG{k}{return} \PYG{n}{redirect}\PYG{p}{(}\PYG{l+s}{\PYGZsq{}}\PYG{l+s}{exercises}\PYG{l+s}{\PYGZsq{}}\PYG{p}{)}

\PYG{c}{\PYGZsh{}Retirer un devoir   }
\PYG{k}{def} \PYG{n+nf}{deleteHomework}\PYG{p}{(}\PYG{n}{request}\PYG{p}{,} \PYG{n}{group\PYGZus{}id}\PYG{p}{,} \PYG{n}{homework\PYGZus{}id}\PYG{p}{)}\PYG{p}{:}
    \PYG{k}{if} \PYG{n}{request}\PYG{o}{.}\PYG{n}{method} \PYG{o}{==} \PYG{l+s}{\PYGZdq{}}\PYG{l+s}{POST}\PYG{l+s}{\PYGZdq{}}\PYG{p}{:}
        
        \PYG{c}{\PYGZsh{}Selon exercice, quiz ou cours}
        \PYG{k}{if} \PYG{l+s}{\PYGZsq{}}\PYG{l+s}{deleteHomeworkEx}\PYG{l+s}{\PYGZsq{}} \PYG{o+ow}{in} \PYG{n}{request}\PYG{o}{.}\PYG{n}{POST}\PYG{p}{:}
            \PYG{n}{exercise} \PYG{o}{=} \PYG{n}{Exercise}\PYG{o}{.}\PYG{n}{objects}\PYG{o}{.}\PYG{n}{filter}\PYG{p}{(}\PYG{n+nb}{id} \PYG{o}{=} \PYG{n}{homework\PYGZus{}id}\PYG{p}{)}
            \PYG{n}{exercise} \PYG{o}{=} \PYG{n}{exercise}\PYG{p}{[}\PYG{l+m+mi}{0}\PYG{p}{]}
            \PYG{n}{group} \PYG{o}{=} \PYG{n}{Group}\PYG{o}{.}\PYG{n}{objects}\PYG{o}{.}\PYG{n}{get}\PYG{p}{(}\PYG{n+nb}{id} \PYG{o}{=} \PYG{n}{group\PYGZus{}id}\PYG{p}{)}
            \PYG{n}{assignedHomework} \PYG{o}{=} \PYG{n}{AssignHomework}\PYG{o}{.}\PYG{n}{objects}\PYG{o}{.}\PYG{n}{filter}\PYG{p}{(}\PYG{n}{group} \PYG{o}{=} \PYG{n}{group}\PYG{p}{,} \PYG{n}{exercise} \PYG{o}{=} \PYG{n}{exercise}\PYG{p}{)}
            \PYG{n}{assignedHomework} \PYG{o}{=} \PYG{n}{assignedHomework}\PYG{p}{[}\PYG{l+m+mi}{0}\PYG{p}{]}
            \PYG{n}{assignedHomework}\PYG{o}{.}\PYG{n}{delete}\PYG{p}{(}\PYG{p}{)}
        \PYG{k}{if} \PYG{l+s}{\PYGZsq{}}\PYG{l+s}{deleteHomeworkQu}\PYG{l+s}{\PYGZsq{}} \PYG{o+ow}{in} \PYG{n}{request}\PYG{o}{.}\PYG{n}{POST}\PYG{p}{:}
            \PYG{n}{quiz} \PYG{o}{=} \PYG{n}{Quiz}\PYG{o}{.}\PYG{n}{objects}\PYG{o}{.}\PYG{n}{filter}\PYG{p}{(}\PYG{n+nb}{id} \PYG{o}{=} \PYG{n}{homework\PYGZus{}id}\PYG{p}{)}
            \PYG{n}{quiz} \PYG{o}{=} \PYG{n}{quiz}\PYG{p}{[}\PYG{l+m+mi}{0}\PYG{p}{]}
            \PYG{n}{group} \PYG{o}{=} \PYG{n}{Group}\PYG{o}{.}\PYG{n}{objects}\PYG{o}{.}\PYG{n}{get}\PYG{p}{(}\PYG{n+nb}{id} \PYG{o}{=} \PYG{n}{group\PYGZus{}id}\PYG{p}{)}
            \PYG{n}{assignedHomework} \PYG{o}{=} \PYG{n}{AssignHomework}\PYG{o}{.}\PYG{n}{objects}\PYG{o}{.}\PYG{n}{filter}\PYG{p}{(}\PYG{n}{group} \PYG{o}{=} \PYG{n}{group}\PYG{p}{,} \PYG{n}{quiz} \PYG{o}{=} \PYG{n}{quiz}\PYG{p}{)}
            \PYG{n}{assignedHomework} \PYG{o}{=} \PYG{n}{assignedHomework}\PYG{p}{[}\PYG{l+m+mi}{0}\PYG{p}{]}
            \PYG{n}{assignedHomework}\PYG{o}{.}\PYG{n}{delete}\PYG{p}{(}\PYG{p}{)}
        \PYG{k}{if} \PYG{l+s}{\PYGZsq{}}\PYG{l+s}{deleteHomeworkCo}\PYG{l+s}{\PYGZsq{}} \PYG{o+ow}{in} \PYG{n}{request}\PYG{o}{.}\PYG{n}{POST}\PYG{p}{:}
            \PYG{n}{course} \PYG{o}{=} \PYG{n}{Course}\PYG{o}{.}\PYG{n}{objects}\PYG{o}{.}\PYG{n}{filter}\PYG{p}{(}\PYG{n+nb}{id} \PYG{o}{=} \PYG{n}{homework\PYGZus{}id}\PYG{p}{)}
            \PYG{n}{course} \PYG{o}{=} \PYG{n}{course}\PYG{p}{[}\PYG{l+m+mi}{0}\PYG{p}{]}
            \PYG{n}{group} \PYG{o}{=} \PYG{n}{Group}\PYG{o}{.}\PYG{n}{objects}\PYG{o}{.}\PYG{n}{get}\PYG{p}{(}\PYG{n+nb}{id} \PYG{o}{=} \PYG{n}{group\PYGZus{}id}\PYG{p}{)}
            \PYG{n}{assignedHomework} \PYG{o}{=} \PYG{n}{AssignHomework}\PYG{o}{.}\PYG{n}{objects}\PYG{o}{.}\PYG{n}{filter}\PYG{p}{(}\PYG{n}{group} \PYG{o}{=} \PYG{n}{group}\PYG{p}{,} \PYG{n}{course} \PYG{o}{=} \PYG{n}{course}\PYG{p}{)}
            \PYG{n}{assignedHomework} \PYG{o}{=} \PYG{n}{assignedHomework}\PYG{p}{[}\PYG{l+m+mi}{0}\PYG{p}{]}
            \PYG{n}{assignedHomework}\PYG{o}{.}\PYG{n}{delete}\PYG{p}{(}\PYG{p}{)}
    \PYG{k}{return} \PYG{n}{redirect}\PYG{p}{(}\PYG{l+s}{\PYGZsq{}}\PYG{l+s}{group\PYGZus{}view}\PYG{l+s}{\PYGZsq{}}\PYG{p}{,} \PYG{n}{group\PYGZus{}id} \PYG{o}{=} \PYG{n}{group\PYGZus{}id}\PYG{p}{)}
\end{Verbatim}


\subsection{\texttt{urls.py}}
\label{annexes:urls-py}
\begin{Verbatim}[commandchars=\\\{\},numbers=left,firstnumber=1,stepnumber=1]
\PYG{k+kn}{from} \PYG{n+nn}{django.conf.urls} \PYG{k+kn}{import} \PYG{n}{patterns}\PYG{p}{,} \PYG{n}{include}\PYG{p}{,} \PYG{n}{url}
\PYG{k+kn}{from} \PYG{n+nn}{django.contrib} \PYG{k+kn}{import} \PYG{n}{admin}
\PYG{k+kn}{from} \PYG{n+nn}{dashboard.views} \PYG{k+kn}{import} \PYG{o}{*}


\PYG{n}{urlpatterns} \PYG{o}{=} \PYG{n}{patterns}\PYG{p}{(}\PYG{l+s}{\PYGZsq{}}\PYG{l+s}{teachers.views}\PYG{l+s}{\PYGZsq{}}\PYG{p}{,}
    \PYG{n}{url}\PYG{p}{(}\PYG{l+s}{r\PYGZsq{}}\PYG{l+s}{\PYGZca{}home/\PYGZdl{}}\PYG{l+s}{\PYGZsq{}}\PYG{p}{,} \PYG{n}{home}\PYG{p}{,} \PYG{n}{name}\PYG{o}{=}\PYG{l+s}{\PYGZsq{}}\PYG{l+s}{home}\PYG{l+s}{\PYGZsq{}}\PYG{p}{)}\PYG{p}{,}
    \PYG{n}{url}\PYG{p}{(}\PYG{l+s}{r\PYGZsq{}}\PYG{l+s}{\PYGZca{}classe/(?P\PYGZlt{}group\PYGZus{}id\PYGZgt{}}\PYG{l+s}{\PYGZbs{}}\PYG{l+s}{d+)/\PYGZdl{}}\PYG{l+s}{\PYGZsq{}}\PYG{p}{,} \PYG{n}{group}\PYG{p}{,} \PYG{n}{name}\PYG{o}{=}\PYG{l+s}{\PYGZsq{}}\PYG{l+s}{group\PYGZus{}view}\PYG{l+s}{\PYGZsq{}}\PYG{p}{)}\PYG{p}{,}
    \PYG{n}{url}\PYG{p}{(}\PYG{l+s}{r\PYGZsq{}}\PYG{l+s}{\PYGZca{}exercices/\PYGZdl{}}\PYG{l+s}{\PYGZsq{}}\PYG{p}{,} \PYG{n}{exercises}\PYG{p}{,} \PYG{n}{name}\PYG{o}{=}\PYG{l+s}{\PYGZsq{}}\PYG{l+s}{exercises}\PYG{l+s}{\PYGZsq{}}\PYG{p}{)}\PYG{p}{,}
    \PYG{n}{url}\PYG{p}{(}\PYG{l+s}{r\PYGZsq{}}\PYG{l+s}{\PYGZca{}nouveau\PYGZus{}groupe/\PYGZdl{}}\PYG{l+s}{\PYGZsq{}}\PYG{p}{,} \PYG{n}{newgroup}\PYG{p}{,} \PYG{n}{name}\PYG{o}{=}\PYG{l+s}{\PYGZsq{}}\PYG{l+s}{newgroup}\PYG{l+s}{\PYGZsq{}}\PYG{p}{)}\PYG{p}{,}
    \PYG{n}{url}\PYG{p}{(}\PYG{l+s}{r\PYGZsq{}}\PYG{l+s}{\PYGZca{}profil/\PYGZdl{}}\PYG{l+s}{\PYGZsq{}}\PYG{p}{,} \PYG{n}{profil}\PYG{p}{,} \PYG{n}{name} \PYG{o}{=} \PYG{l+s}{\PYGZsq{}}\PYG{l+s}{profil}\PYG{l+s}{\PYGZsq{}}\PYG{p}{)}\PYG{p}{,}
    \PYG{c}{\PYGZsh{}Pour retirer d\PYGZsq{}un groupe}
    \PYG{n}{url}\PYG{p}{(}\PYG{l+s}{r\PYGZsq{}}\PYG{l+s}{\PYGZca{}enlever\PYGZus{}groupe/(?P\PYGZlt{}group\PYGZus{}id\PYGZgt{}}\PYG{l+s}{\PYGZbs{}}\PYG{l+s}{d+)/(?P\PYGZlt{}member\PYGZus{}id\PYGZgt{}}\PYG{l+s}{\PYGZbs{}}\PYG{l+s}{d+)/\PYGZdl{}}\PYG{l+s}{\PYGZsq{}}\PYG{p}{,}
        \PYG{n}{deleteFromGroup}\PYG{p}{,} \PYG{n}{name} \PYG{o}{=} \PYG{l+s}{\PYGZdq{}}\PYG{l+s}{deleteFromGroup}\PYG{l+s}{\PYGZdq{}}\PYG{p}{)}\PYG{p}{,}
    \PYG{c}{\PYGZsh{}Pour supprimer une activité}
    \PYG{n}{url}\PYG{p}{(}\PYG{l+s}{r\PYGZsq{}}\PYG{l+s}{\PYGZca{}enlever\PYGZus{}activité/(?P\PYGZlt{}activity\PYGZus{}id\PYGZgt{}}\PYG{l+s}{\PYGZbs{}}\PYG{l+s}{d+)/\PYGZdl{}}\PYG{l+s}{\PYGZsq{}}\PYG{p}{,} 
        \PYG{n}{deleteActivity}\PYG{p}{,} \PYG{n}{name} \PYG{o}{=} \PYG{l+s}{\PYGZdq{}}\PYG{l+s}{deleteActivity}\PYG{l+s}{\PYGZdq{}}\PYG{p}{)}\PYG{p}{,}
    \PYG{c}{\PYGZsh{}Pour retirer un devoir}
    \PYG{n}{url}\PYG{p}{(}\PYG{l+s}{r\PYGZsq{}}\PYG{l+s}{\PYGZca{}enlever\PYGZus{}devoir/(?P\PYGZlt{}group\PYGZus{}id\PYGZgt{}}\PYG{l+s}{\PYGZbs{}}\PYG{l+s}{d+)/(?P\PYGZlt{}homework\PYGZus{}id\PYGZgt{}}\PYG{l+s}{\PYGZbs{}}\PYG{l+s}{d+)/\PYGZdl{}}\PYG{l+s}{\PYGZsq{}}\PYG{p}{,}
        \PYG{n}{deleteHomework}\PYG{p}{,} \PYG{n}{name} \PYG{o}{=} \PYG{l+s}{\PYGZdq{}}\PYG{l+s}{deleteHomework}\PYG{l+s}{\PYGZdq{}}\PYG{p}{)}\PYG{p}{,}
\PYG{p}{)}
\end{Verbatim}


\chapter{Table des illustrations}
\label{illu::doc}\label{illu:table-des-illustrations}\begingroup
\let\clearpage\relax
\renewcommand*\listfigurename{Table des illustrations}
\listoffigures
\endgroup


\renewcommand{\indexname}{Index}
\printindex
\end{document}
